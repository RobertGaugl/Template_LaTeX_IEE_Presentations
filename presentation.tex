\documentclass[aspectratio=169]{beamer}

%% -------------------------------------------------------------------
%% Theme Configuration
%% -------------------------------------------------------------------
\usetheme[]{iee}


%% -------------------------------------------------------------------
%% Packages
%% -------------------------------------------------------------------
\usepackage{fontawesome5}
\usepackage{pgfplots}
\usepackage{pgf-pie}
\pgfplotsset{compat=1.18} % or your current pgfplots version
\usepackage[table]{xcolor}
\usepackage{csquotes} % Recommended with babel + biblatex
\usepackage{tabularray}
\UseTblrLibrary{booktabs} % Optional, for better lines in tables
\UseTblrLibrary{siunitx}

\usepackage[style=ieee,backend=biber]{biblatex} % Bibliography
\addbibresource{references.bib}


%% -------------------------------------------------------------------
%% Metadata
%% -------------------------------------------------------------------
% Select language
\usepackage[english]{babel}
% \usepackage[ngerman]{babel}

\title[Short Title]{IEE Template}
%\twolinetitletrue % Activate if title is two lines long
\subtitle{for \LaTeX~Presentations}

\author{Robert Gaugl}
\newcommand{\academictitlesauthor}{Ass.Prof. Dipl.Ing. Dr.techn.}
\newcommand{\coauthors}{}

\date{Presentation Event/Date}

\newcommand{\university}{Graz University of Technology}
\institute{Institute of Electricity Economics and Energy Innovation}
\newcommand{\street}{Inffeldgasse 18}
\newcommand{\postalcode}{8020}
\newcommand{\city}{Graz}
\newcommand{\country}{Austria}

\newcommand{\phone}{+43 316 873 7904}
\newcommand{\mailauthor}{robert.gaugl@tugraz.at}
\newcommand{\urlinstitute}{iee.tugraz.at} % Without https://

\newcommand{\urlinstagram}{instagram.com/iee.tugraz} % Without https://
\newcommand{\urllinkedin}{instagram.com/iee.tugraz} % Without https://

\logobar{Supported by: Test} % titlepage sponsors

%% -------------------------------------------------------------------
%% Formatting
%% -------------------------------------------------------------------

\germanenglish{%
    \sisetup{
        output-decimal-marker = {,},  % use comma as decimal separator
        group-separator = {.},        % use dot as thousand separator
        group-minimum-digits = 4,     % format large numbers
        group-digits=integer,
        detect-mode = true,
        detect-family = false,
    }         
    }{
    \sisetup{
        output-decimal-marker = {.},  % use comma as decimal separator
        group-separator = {~},        % use space as thousand separator
        group-minimum-digits = 4,     % optional: format large numbers
        group-digits=integer,
        detect-mode = true,
        detect-family = false,
    }
}
\DeclareSIUnit \watthour { Wh } %apparent power 

%% -------------------------------------------------------------------
%% Document
%% -------------------------------------------------------------------
\begin{document}

%% -------------------------------------------------------------------
%% Title Slide
\begin{frame}[plain]
    \maketitleslide
\end{frame}


%% -------------------------------------------------------------------
%% Section: Understand the Template for a Consistent Layout
\section{Understand the \textbf{Template} for a Consistent Layout}

\begin{frame}
    \agenda{Agenda}{} % No subtitle and highlights currentvsection
    %\agenda[sections=all]{Agenda}{Overview of lecture structure} % With subtitle and all sections highlighted
\end{frame}


\begin{frame}{Template}
    \framesubtitle{Footnote and Language}

    % Insert a Faded picture (ALWAYS FIRST OR LAST THING IN A FRAME)
    \insertfadedpicture{15.98cm}{figures/iee_besprechung.png}{Source: Institute of Electricity Economics and Energy Innovation/TU Graz}
    
    \begin{minipage}{0.8\textwidth}  % 80% of the slide width
        \begin{itemize}
            \item To \textcolor{yellow}{\textbf{change}} text/name in \textcolor{yellow}{\textbf{Footnote}} and \textcolor{yellow}{\textbf{Date on Title Slide}}: 
            \vspace{-0.5\topsep}
            \begin{itemize}
                \item The presentation.tex file contains a section at the top labeled Metadata
                \item Modifying the author, date, or institute in this section will automatically update the footnote and date displayed in the presentation
            \end{itemize}
    
            \item Footnote Text
            \vspace{-0.5\topsep}
            \begin{itemize}
                \item The \textcolor{green}{\textbf{default footnote}} text displays the \textcolor{green}{\textbf{institute's name}}
                \item If you want to include your \textcolor{green}{\textbf{lecture title}}, \textcolor{green}{\textbf{presentation topic}}, or other context-specific information in the \textcolor{green}{\textbf{footnote}}, simply enter it in the \textcolor{green}{\textbf{institute field}} within the metadata section
            \end{itemize}
            
            \item Language
            \vspace{-0.5\topsep}
            \begin{itemize}
                \item Ensure that the \textcolor{iee}{\textbf{language}} used throughout the presentation \textcolor{iee}{\textbf{remains consistent}}
                \item For presentations in \textcolor{iee}{\textbf{German}}, set \textcolor{iee}{\textbf{language to DE}} in the metadata section.
                \item The \textcolor{iee}{\textbf{footnote text}} must \textcolor{iee}{\textbf{match}} the \textcolor{iee}{\textbf{language}} of the presentation
            \end{itemize}
        \end{itemize}
    \end{minipage}
\end{frame}

\begin{frame}{Color palette}
    \framesubtitle{Make use of the the Defined Colors}
    \vspace{1cm}
    \begin{columns}[t]
        \column{.25\textwidth}
            \begin{beamercolorbox}[center,colsep*=4pt]{white}\textbf{Light Shades}\strut\end{beamercolorbox}
            \begin{beamercolorbox}[center,colsep*=4pt]{white}\strut\end{beamercolorbox}
            \begin{beamercolorbox}[center,colsep*=4pt]{greyLight}greenLight\strut\end{beamercolorbox}
            \begin{beamercolorbox}[center,colsep*=4pt]{blueLight}blueLight\strut\end{beamercolorbox}
            \begin{beamercolorbox}[center,colsep*=4pt]{ieeLight}ieeLight\strut\end{beamercolorbox}
            \begin{beamercolorbox}[center,colsep*=4pt]{redLight}redLight\strut\end{beamercolorbox}
            \begin{beamercolorbox}[center,colsep*=4pt]{turquoiseLight}turquoiseLight\strut\end{beamercolorbox}
            \begin{beamercolorbox}[center,colsep*=4pt]{greenLight}greenLight\strut\end{beamercolorbox}
            \begin{beamercolorbox}[center,colsep*=4pt]{yellowLight}yellowLight\strut\end{beamercolorbox}
        \column{.25\textwidth}
            \begin{beamercolorbox}[center,colsep*=4pt]{white}\textbf{Standard}\strut\end{beamercolorbox}
            \begin{beamercolorbox}[center,colsep*=4pt]{white}\strut\end{beamercolorbox}
            \begin{beamercolorbox}[center,colsep*=4pt]{grey}grey\strut\end{beamercolorbox}
            \begin{beamercolorbox}[center,colsep*=4pt]{blue}blue\strut\end{beamercolorbox}
            \begin{beamercolorbox}[center,colsep*=4pt]{iee}iee\strut\end{beamercolorbox}
            \begin{beamercolorbox}[center,colsep*=4pt]{red}red\strut\end{beamercolorbox}
            \begin{beamercolorbox}[center,colsep*=4pt]{turquoise}turquoise\strut\end{beamercolorbox}
            \begin{beamercolorbox}[center,colsep*=4pt]{green}green\strut\end{beamercolorbox}
            \begin{beamercolorbox}[center,colsep*=4pt]{yellow}yellow\strut\end{beamercolorbox}
        \column{.25\textwidth}
            \begin{beamercolorbox}[center,colsep*=4pt]{white}\textbf{Dark Shades}\strut\end{beamercolorbox}
            \begin{beamercolorbox}[center,colsep*=4pt]{white}\strut\end{beamercolorbox}
            \begin{beamercolorbox}[center,colsep*=4pt]{greyDark}greyDark\strut\end{beamercolorbox}
            \begin{beamercolorbox}[center,colsep*=4pt]{blueDark}blueDark\strut\end{beamercolorbox}
            \begin{beamercolorbox}[center,colsep*=4pt]{ieeDark}ieeDark\strut\end{beamercolorbox}
            \begin{beamercolorbox}[center,colsep*=4pt]
            {redDark}redDark\strut\end{beamercolorbox}
            \begin{beamercolorbox}[center,colsep*=4pt]{turquoiseDark}turquoiseDark\strut\end{beamercolorbox}
            \begin{beamercolorbox}[center,colsep*=4pt]{greenDark}greenDark\strut\end{beamercolorbox}
            \begin{beamercolorbox}[center,colsep*=4pt]{yellowDark}yellowDark\strut\end{beamercolorbox}
    \end{columns}
\end{frame}

%% -------------------------------------------------------------------
%% Leverage Boxes to Organize Content
\section{Leverage \textbf{Boxes} to Organize Content}

\begin{frame}
    \agenda{Agenda}{} % No subtitle and highlights currentvsection
    %\agenda[sections=all]{Agenda}{Overview of lecture structure} % With subtitle and all sections highlighted
\end{frame}


\begin{frame}{Boxes}
    \framesubtitle{Organize your Content}
    \begin{coloredblock}[yellow]
        \begin{itemize}
            \item Instead of using lists or enumerations, consider organizing your content into distinct groups for better structure
            \item Boxes are an excellent tool to enhance visual clarity and emphasize group distinctions
        \end{itemize}
    \end{coloredblock}

    % Bottom blocks (side-by-side)
    \begin{minipage}[t]{0.49\textwidth}
        \begin{coloredblock}[blue][\centering\faIcon{palette}~~~Designing Boxes][][6cm]
                \begin{itemize}
                    \item \textbf{Title Bar}: You can choose between boxes with or without a title bar.
                    \item \textbf{Itemization}: Creating itemized lists inside a box requires a bit of extra effort, but this example shows how it's done.
                \end{itemize}
        \end{coloredblock}
    \end{minipage}
    \hfill
    \begin{minipage}[t]{0.49\textwidth}
        \begin{coloredblock}[blue][\centering\faIcon{lightbulb}~~~Additional Tips][][6cm]
            \begin{itemize}
                \item Use \textbf{short}, concise \textbf{text} inside the box to maximize clarity
                \item Consider \textbf{icons} in the title bar to enhance visual appeal. You can use the icons from the fontawesome package.
            \end{itemize}
        \end{coloredblock}
    \end{minipage}
\end{frame}

\begin{frame}{Boxes}
    \framesubtitle{Standard Boxes with Title}

    \footnotesize\centering\texttt{\textbackslash begin\{coloredblock\}[color][Optional:~Title][Optional:~titlesize][Optional:~Height cm]}
    
    \begin{minipage}[t]{0.49\textwidth}
    
        \begin{coloredblock}[blue][Blue Block with Title]
            \texttt{\textbackslash begin\{coloredblock\}[blue][Blue Block with Title]}\strut
        \end{coloredblock}

        \begin{coloredblock}[yellow][Yellow Block with Title]
            \texttt{\textbackslash begin\{coloredblock\}[yellow][Yellow Block with Title]}\strut
        \end{coloredblock}

        \begin{coloredblock}[red][Red Block with Title]
            \texttt{\textbackslash begin\{coloredblock\}[red][Red Block with Title]}\strut
        \end{coloredblock}
        
    \end{minipage}
    \hfill
    \begin{minipage}[t]{0.49\textwidth}
    
        \begin{coloredblock}[iee][IEE Block with Title]
            \texttt{\textbackslash begin\{coloredblock\}[iee][IEE Block with Title]}\strut
        \end{coloredblock}

        \begin{coloredblock}[green][Green Block with Title]
            \texttt{\textbackslash begin\{coloredblock\}[green][Green Block with Title]}\strut
        \end{coloredblock}

        \begin{coloredblock}[grey][Grey Block with Title]
            \texttt{\textbackslash begin\{coloredblock\}[grey][Grey Block with Title]}\strut
        \end{coloredblock}
        
    \end{minipage}
\end{frame}

\begin{frame}{Boxes}
    \framesubtitle{Standard Boxes without Title}

    \footnotesize\centering\texttt{\textbackslash begin\{coloredblock\}[color][Optional:~Title][Optional:~titlesize][Optional:~Height cm]}
    
    \begin{minipage}[t]{0.49\textwidth}

        \begin{coloredblock}[blue]
            \texttt{\textbackslash begin\{coloredblock\}[blue]}\strut
        \end{coloredblock}

        \begin{coloredblock}[yellow]
            \texttt{\textbackslash begin\{coloredblock\}[yellow]}\strut
        \end{coloredblock}

        \begin{coloredblock}[red]
            \texttt{\textbackslash begin\{coloredblock\}[red]}\strut
        \end{coloredblock}

    \end{minipage}
    \hfill
    \begin{minipage}[t]{0.49\textwidth}
        
        \begin{coloredblock}[iee]
            \texttt{\textbackslash begin\{coloredblock\}[iee]}\strut
        \end{coloredblock}

        \begin{coloredblock}[green]
            \texttt{\textbackslash begin\{coloredblock\}[green]}\strut
        \end{coloredblock}

        \begin{coloredblock}[grey]
            \texttt{\textbackslash begin\{coloredblock\}[grey]}\strut
        \end{coloredblock}
        
    \end{minipage}

    \centering
    \begin{minipage}[t]{0.49\textwidth}
        \begin{coloredblock}[turquoise]
                \texttt{\textbackslash begin\{coloredblock\}[turquoise]}\strut
        \end{coloredblock}
    \end{minipage}
\end{frame}


\begin{frame}{Boxes}
    \framesubtitle{Dark Boxes with Title}

    \footnotesize\centering\texttt{\textbackslash begin\{coloredblockdark\}[color][Optional:~Title][Optional:~titlesize][Optional:~Height cm]}
    
    \begin{minipage}[t]{0.49\textwidth}
    
        \begin{coloredblockdark}[blue][Dark Blue Block with Title]
            \texttt{\textbackslash begin\{coloredblockdark\}[blue][Blue Block with Title]}\strut
        \end{coloredblockdark}

        \begin{coloredblockdark}[yellow][Dark Yellow Block with Title]
            \texttt{\textbackslash begin\{coloredblockdark\}[yellow]\allowbreak [Yellow Block with Title]}\strut
        \end{coloredblockdark}

        \begin{coloredblockdark}[red][Dark Red Block with Title]
            \texttt{\textbackslash begin\{coloredblockdark\}[red][Red Block with Title]}\strut
        \end{coloredblockdark}
        
    \end{minipage}
    \hfill
    \begin{minipage}[t]{0.49\textwidth}
    
        \begin{coloredblockdark}[iee][Dark IEE Block with Title]
            \texttt{\textbackslash begin\{coloredblockdark\}[iee][IEE Block with Title]}\strut
        \end{coloredblockdark}

        \begin{coloredblockdark}[green][Dark Green Block with Title]
            \texttt{\textbackslash begin\{coloredblockdark\}[green]\allowbreak [Green Block with Title]}\strut
        \end{coloredblockdark}

        \begin{coloredblockdark}[grey][Dark Grey Block with Title]
            \texttt{\textbackslash begin\{coloredblockdark\}[grey][Grey Block with Title]}\strut
        \end{coloredblockdark}
        
    \end{minipage}
\end{frame}

\begin{frame}{Boxes}
    \framesubtitle{Dark Boxes without Title}

    \footnotesize\centering\texttt{\textbackslash begin\{coloredblockdark\}[color][Optional:~Title][Optional:~titlesize][Optional:~Height cm]}
    
    \begin{minipage}[t]{0.49\textwidth}

        \begin{coloredblockdark}[blue]
            \texttt{\textbackslash begin\{coloredblockdark\}[blue]}\strut
        \end{coloredblockdark}

        \begin{coloredblockdark}[yellow]
            \texttt{\textbackslash begin\{coloredblockdark\}[yellow]}\strut
        \end{coloredblockdark}

        \begin{coloredblockdark}[red]
            \texttt{\textbackslash begin\{coloredblockdark\}[red]}\strut
        \end{coloredblockdark}

    \end{minipage}
    \hfill
    \begin{minipage}[t]{0.49\textwidth}
        
        \begin{coloredblockdark}[iee]
            \texttt{\textbackslash begin\{coloredblockdark\}[iee]}\strut
        \end{coloredblockdark}

        \begin{coloredblockdark}[green]
            \texttt{\textbackslash begin\{coloredblockdark\}[green]}\strut
        \end{coloredblockdark}

        \begin{coloredblockdark}[grey]
            \texttt{\textbackslash begin\{coloredblockdark\}[grey]}\strut
        \end{coloredblockdark}
        
    \end{minipage}

    \centering
    \begin{minipage}[t]{0.49\textwidth}
        \begin{coloredblockdark}[turquoise]
                \texttt{\textbackslash begin\{coloredblockdark\}[turquoise]}\strut
        \end{coloredblockdark}
        
    \end{minipage}
\end{frame}


%% -------------------------------------------------------------------
%% Use the Typeface to Enhance Readability
\section{Use the \textbf{Typeface} to Enhance Readability}

\begin{frame}
    \agenda{Agenda}{} % No subtitle and highlights currentvsection
    %\agenda[sections=all]{Agenda}{Overview of lecture structure} % With subtitle and all sections highlighted
\end{frame}

\begin{frame}{Typeface}
    \framesubtitle{Font and Styling}

    \begin{coloredblock}[blue][\centering\faIcon{highlighter}~~~Highlighting]
        \begin{itemize}
            \item Use \textbf{bold text} to \textbf{emphasize} key points
            \item For a subtler emphasis, you can use \textit{italics}
            \item Avoid using \underline{underlines} for emphasis, as underlined text is commonly interpreted as a hyperlink and may lead to confusion
        \end{itemize}
    \end{coloredblock}

    \begin{coloredblock}[yellow][\centering\faIcon{fill-drip}~~~Color]
            \begin{itemize}
                \item Make the most of the \textbf{default colors} available (IEE, blue, yellow, green, red and grey)
                \item To \textbf{color text}, use \texttt{\textbackslash textcolor\{color\{Text\}\}}
                \item To draw attention, use \textbf{yellow} for \textbf{moderate emphasis} and \textbf{red} for \textbf{strong emphasis}
            \end{itemize}
    \end{coloredblock}
    
\end{frame}

\begin{frame}{Typeface}
    \framesubtitle{Available Font Sizes}
    \vfill
    \begin{columns}
        \column{.3\textwidth}%
        \centering
        \thefontsize[TINY]\TINY
        \thefontsize[Tiny]\Tiny
        \thefontsize[myFootnotesize]\myFootnotesize
        \thefontsize[tiny]\tiny
        \thefontsize[scriptsize]\scriptsize
        \thefontsize[footnotesize]\footnotesize
        \thefontsize[small]\small
        \thefontsize[normalsize]\normalsize
        \thefontsize[large]\large
    
        \column{.65\textwidth}%
        \centering
        \thefontsize[Large]\Large
        \thefontsize[LARGE]\LARGE
        \thefontsize[huge]\huge
        \thefontsize[Huge]\Huge
    \end{columns}
    \vfill
\end{frame}

%% -------------------------------------------------------------------
%% Visualize your Numbers with Plots
\section{Visualize your Numbers with \textbf{Plots}}

\begin{frame}
    \agenda{Agenda}{} % No subtitle and highlights currentvsection
    %\agenda[sections=all]{Agenda}{Overview of lecture structure} % With subtitle and all sections highlighted
\end{frame}

\begin{frame}{Plots}
    \framesubtitle{Visualize your Numbers}
    \begin{minipage}[t]{0.49\textwidth}
        \vspace{-1cm}
        
        \begin{coloredblock}[blue][][][1.5cm]
            \makebox[\textwidth]{%
                \begin{minipage}[c][1.5cm][c]{0.1\textwidth}
                    \centering
                    \textcolor{blueDark}{\faIcon{chart-line}}
                \end{minipage}%
                \hfill
                \begin{minipage}[c][1.5cm][c]{0.89\textwidth}
                    \footnotesize
                Set \textbf{axis line style} to \textbf{black} to add a black \textbf{border} around the plot area
                \end{minipage}%
            }
        \end{coloredblock}

        \begin{coloredblock}[yellow][][][1.5cm]
            \makebox[\textwidth]{%
                \begin{minipage}[c][1.5cm][c]{0.1\textwidth}
                    \centering
                    \textcolor{yellowDark}{\faIcon{list-ol}}
                \end{minipage}%
                \hfill
                \begin{minipage}[c][1.5cm][c]{0.89\textwidth}
                    \footnotesize
                Include \textbf{axis label}s with descriptions and \textbf{units} (e.g., in Unit) \& Thousand Separator
                \end{minipage}%
            }
        \end{coloredblock}
   
        \begin{coloredblock}[grey][][][1.5cm]
            \makebox[\textwidth]{%
                \begin{minipage}[c][1.5cm][c]{0.1\textwidth}
                    \centering
                    \textcolor{greyDark}{\faIcon[regular]{square}}
                \end{minipage}%
                \hfill
                \begin{minipage}[c][1.5cm][c]{0.89\textwidth}
                    \footnotesize
                Add a \textbf{black border} around the \textbf{sections} of the bar chart
                \end{minipage}%
            }
        \end{coloredblock}

        \begin{coloredblock}[green][][][1.5cm]
            \makebox[\textwidth]{%
                \begin{minipage}[c][1.5cm][c]{0.1\textwidth}
                    \centering
                    \textcolor{greenDark}{\faIcon{chart-bar}}
                \end{minipage}%
                \hfill
                \begin{minipage}[c][1.5cm][c]{0.89\textwidth}
                    \footnotesize
                    Include a \textbf{legend} to explain the chart's data categories
                \end{minipage}%
            }
        \end{coloredblock}
        
        \begin{coloredblock}[iee][\centering \faIcon{star}~~~Optional Improvements][\footnotesize][3cm]
                \begin{itemize}
                    \item \footnotesize Add grey grid lines to the background
                    \item \footnotesize Include caption in italics at the bottom, complete with figure numbering
                \end{itemize}
        \end{coloredblock}
    \end{minipage}
    \hfill
    % Left side of the page
    \begin{minipage}[t]{0.49\textwidth}
        \vspace{0.1cm}
        \tiny
        \begin{figure}[htbp]
            \begin{tikzpicture}
                \begin{axis}[
                    ybar stacked,
                    bar width=45pt,
                    width=\linewidth,  % Use full minipage width
                    height=12cm,        % Adjust to fit nicely
                    enlarge x limits=0.15,
                    ymin=0,
                    ylabel={Installed Capacity in MW},
                    xlabel={Year},
                    symbolic x coords={2022,2023,2024,2025},
                    xtick=data,
                    xmajorgrids=false,
                    ymajorgrids=true,
                    major grid style={gray!30},
                    axis line style={black},
                    tick style={draw=none},
                    legend style={at={(0.5,-0.15)}, anchor=north, legend columns=-1, font=\tiny},
                    area legend,
                ]
                ]
                \addplot+[draw=black, fill=coalColor] coordinates {(2022,25) (2023,20) (2024,15) (2025,10)};
                \addplot+[draw=black, fill=gasColor] coordinates {(2022,75) (2023,70) (2024,60) (2025,55)};
                \addplot+[draw=black, fill=rorColor] coordinates {(2022,40) (2023,45) (2024,50) (2025,50)};
                \addplot+[draw=black, fill=pvColor] coordinates {(2022,50) (2023,70) (2024,90) (2025,120)};
                \addplot+[draw=black, fill=windColor] coordinates {(2022,80) (2023,90) (2024,110) (2025,130)};
                \addplot+[draw=black, fill=storageHydroColor] coordinates {(2022,10) (2023,20) (2024,30) (2025,40)};
                \legend{Coal, Gas, Run-of-River, PV, Wind, Storage}
                \end{axis}
            \end{tikzpicture}
            \caption{\centering Installed capacity per power plant type in Wakanda from 2022-2025.}
            \label{fig:installed_capacity}
        \end{figure}
    \end{minipage}
    \begin{tikzpicture}[remember picture, overlay]
     \node[anchor=south east, xshift=-0.5cm, yshift=0.6cm] at (current page.south east) {
       {\myFootnotesize Source: Energy Report, Statistik Wakanda, 2023}
     };
   \end{tikzpicture}
\end{frame}


\begin{frame}{Plots}
    \framesubtitle{Visualize your Numbers}
    \begin{minipage}[t]{0.49\textwidth}
        \vspace{-1cm}
        
        \begin{coloredblock}[blue][][][1.5cm]
            \makebox[\textwidth]{%
                \begin{minipage}[c][1.5cm][c]{0.1\textwidth}
                    \centering
                    \textcolor{blueDark}{\faIcon{chart-line}}
                \end{minipage}%
                \hfill
                \begin{minipage}[c][1.5cm][c]{0.89\textwidth}
                    \footnotesize
                Set \textbf{axis line style} to \textbf{black} to add a black \textbf{border} around the plot area
                \end{minipage}%
            }
        \end{coloredblock}

        \begin{coloredblock}[yellow][][][1.5cm]
            \makebox[\textwidth]{%
                \begin{minipage}[c][1.5cm][c]{0.1\textwidth}
                    \centering
                    \textcolor{yellowDark}{\faIcon{list-ol}}
                \end{minipage}%
                \hfill
                \begin{minipage}[c][1.5cm][c]{0.89\textwidth}
                    \footnotesize
                Include \textbf{axis label}s with descriptions and \textbf{units} (e.g., in Unit) \& Thousand Separator
                \end{minipage}%
            }
        \end{coloredblock}
   
        \begin{coloredblock}[grey][][][1.5cm]
            \makebox[\textwidth]{%
                \begin{minipage}[c][1.5cm][c]{0.1\textwidth}
                    \centering
                    \textcolor{greyDark}{\faIcon[regular]{square}}
                \end{minipage}%
                \hfill
                \begin{minipage}[c][1.5cm][c]{0.89\textwidth}
                    \footnotesize
                Add a \textbf{black border} around the \textbf{sections} of the bar chart
                \end{minipage}%
            }
        \end{coloredblock}

        \begin{coloredblock}[green][][][1.5cm]
            \makebox[\textwidth]{%
                \begin{minipage}[c][1.5cm][c]{0.1\textwidth}
                    \centering
                    \textcolor{greenDark}{\faIcon{chart-bar}}
                \end{minipage}%
                \hfill
                \begin{minipage}[c][1.5cm][c]{0.89\textwidth}
                    \footnotesize
                    Include a \textbf{legend} to explain the chart's data categories
                \end{minipage}%
            }
        \end{coloredblock}
        
        \begin{coloredblock}[iee][\centering \faIcon{star}~~~Optional Improvements][\footnotesize][3cm]
                \begin{itemize}
                    \item \footnotesize Add grey grid lines to the background
                    \item \footnotesize Include caption in italics at the bottom, complete with figure numbering
                \end{itemize}
        \end{coloredblock}
    \end{minipage}
    \hfill
    % Left side of the page
    \begin{minipage}[t]{0.49\textwidth}
        \vspace{0.1cm}
        \tiny
        \begin{figure}[htbp]
            \begin{tikzpicture}
                \begin{axis}[
                    width=\linewidth,
                    height=12cm,
                    xlabel={Hour of Day},
                    ylabel={Demand in MW},
                    xmin=0, xmax=23,
                    ymin=0, ymax=150,
                    xtick={0,4,8,12,16,20,23},
                    ymajorgrids=true,
                    grid style={gray!30},
                    legend style={at={(0.5,-0.2)}, anchor=north, legend columns=3},
                    tick style={draw=none},
                    axis line style={black},
                    legend cell align={left},
                ]
            
                % Sample data for Day 1
                \addplot+[mark=none, very thick, color=default] coordinates {
                    (0,40) (1,38) (2,35) (3,34) (4,36) (5,42)
                    (6,55) (7,70) (8,90) (9,100) (10,105) (11,110)
                    (12,108) (13,105) (14,100) (15,98) (16,102) (17,110)
                    (18,115) (19,120) (20,110) (21,90) (22,70) (23,50)
                };
        
                % Sample data for Day 2
                \addplot+[mark=none, very thick, color=green] coordinates {
                    (0,45) (1,43) (2,40) (3,38) (4,40) (5,48)
                    (6,60) (7,75) (8,95) (9,110) (10,115) (11,120)
                    (12,118) (13,115) (14,110) (15,108) (16,115) (17,120)
                    (18,125) (19,130) (20,120) (21,100) (22,80) (23,60)
                };
        
                % Sample data for Day 3
                \addplot+[mark=none, very thick, color=yellow] coordinates {
                    (0,30) (1,28) (2,25) (3,24) (4,25) (5,32)
                    (6,45) (7,60) (8,80) (9,90) (10,95) (11,100)
                    (12,98) (13,95) (14,90) (15,88) (16,95) (17,100)
                    (18,105) (19,110) (20,100) (21,80) (22,60) (23,40)
                };

                \legend{RP01, RP02, RP03}
                \end{axis}
            \end{tikzpicture}
            \caption{\centering Demand per representative periode in Wakanda.}
            \label{fig:demand}
        \end{figure}
    \end{minipage}
    \begin{tikzpicture}[remember picture, overlay]
     \node[anchor=south east, xshift=-0.5cm, yshift=0.6cm] at (current page.south east) {
       {\myFootnotesize Source: Energy Report, Statistik Wakanda, 2023}
     };
   \end{tikzpicture}
\end{frame}

\begin{frame}{Plots}
    \framesubtitle{Visualize your Numbers}
    \begin{minipage}[t]{0.49\textwidth}
        \vspace{-1cm}
        
        \begin{coloredblock}[blue][][][1.5cm]
            \makebox[\textwidth]{%
                \begin{minipage}[c][1.5cm][c]{0.1\textwidth}
                    \centering
                    \textcolor{blueDark}{\faIcon{chart-line}}
                \end{minipage}%
                \hfill
                \begin{minipage}[c][1.5cm][c]{0.89\textwidth}
                    \footnotesize
                Set \textbf{axis line style} to \textbf{black} to add a black \textbf{border} around the plot area
                \end{minipage}%
            }
        \end{coloredblock}

        \begin{coloredblock}[yellow][][][1.5cm]
            \makebox[\textwidth]{%
                \begin{minipage}[c][1.5cm][c]{0.1\textwidth}
                    \centering
                    \textcolor{yellowDark}{\faIcon{list-ol}}
                \end{minipage}%
                \hfill
                \begin{minipage}[c][1.5cm][c]{0.89\textwidth}
                    \footnotesize
                Include \textbf{axis label}s with descriptions and \textbf{units} (e.g., in Unit) \& Thousand Separator
                \end{minipage}%
            }
        \end{coloredblock}
   
        \begin{coloredblock}[grey][][][1.5cm]
            \makebox[\textwidth]{%
                \begin{minipage}[c][1.5cm][c]{0.1\textwidth}
                    \centering
                    \textcolor{greyDark}{\faIcon[regular]{square}}
                \end{minipage}%
                \hfill
                \begin{minipage}[c][1.5cm][c]{0.89\textwidth}
                    \footnotesize
                Add a \textbf{black border} around the \textbf{sections} of the bar chart
                \end{minipage}%
            }
        \end{coloredblock}

        \begin{coloredblock}[green][][][1.5cm]
            \makebox[\textwidth]{%
                \begin{minipage}[c][1.5cm][c]{0.1\textwidth}
                    \centering
                    \textcolor{greenDark}{\faIcon{chart-bar}}
                \end{minipage}%
                \hfill
                \begin{minipage}[c][1.5cm][c]{0.89\textwidth}
                    \footnotesize
                    Include a \textbf{legend} to explain the chart's data categories
                \end{minipage}%
            }
        \end{coloredblock}
        
        \begin{coloredblock}[iee][\centering \faIcon{star}~~~Optional Improvements][\footnotesize][3cm]
                \begin{itemize}
                    \item \footnotesize Add grey grid lines to the background
                    \item \footnotesize Include caption in italics at the bottom, complete with figure numbering
                \end{itemize}
        \end{coloredblock}
    \end{minipage}
    \hfill
    % Left side of the page
    \begin{minipage}[t]{0.49\textwidth}
        \vspace{0.1cm}
        \tiny
        \begin{figure}[htbp]
            \begin{tikzpicture}
            % Draw the donut chart
            \pie[
                radius=5,
                color={coalColor, gasColor, rorColor, pvColor, windColor, storageHydroColor},
                after number=\%,
                sum=auto,
                text=legend
                %text=none % No labels inside pie
            ]{
                20/Coal,
                70/Gas,
                45/Run-of-River,
                70/PV,
                90/Wind,
                20/Storage
            }

            % Inner white circle to simulate donut hole
            \fill[white] (0,0) circle (2);
            \draw[black, thick] (0,0) circle (2);
            % Text in the center of the donut hole
            \node at (0,0.3) {\scriptsize 315~MW};
            \node at (0,-0.35) {\tiny \textbf{2023}};
        \end{tikzpicture}
            \caption{\centering Shares of installed capacity per power plant type in Wakanda 2023.}
            \label{fig:installed_capacity_2023}
        \end{figure}
    \end{minipage}
    \begin{tikzpicture}[remember picture, overlay]
     \node[anchor=south east, xshift=-0.5cm, yshift=0.6cm] at (current page.south east) {
       {\myFootnotesize Source: Energy Report, Statistik Wakanda, 2023}
     };
   \end{tikzpicture}
\end{frame}


\begin{frame}{Plots}
    \framesubtitle{Colors for Power Plants}

        \begin{table}[htbp]
            \scalebox{0.94}{
            \footnotesize
            \centering
            \begin{tblr}{colspec={lllrrrrc}, % alignement inside cell (l, c r)
                         hline{1-2}={solid,1.5pt}, % Header with black lines on top and bottom
                         hline{3-Y}={grey}, % Grey lines between rows
                         hline{Z}={solid,1.5pt}, % Black line at the end
                         row{odd}={grey!20}, % Background color odd rows
                         % row{even}={white}, % Background color even rows
                         row{1}={white,font=\bfseries}, % Bold font for header
                         cell{2}{8}={bg=oilColor},
                         cell{3}{8}={bg=coalColor},
                         cell{4}{8}={bg=gasColor},
                         cell{5}{8}={bg=otherNonResColor},
                         cell{6}{8}={bg=nuclearColor},
                         cell{7}{8}={bg=biomassColor},
                         cell{8}{8}={bg=rorColor},
                         cell{9}{8}={bg=pvColor},
                         cell{10}{8}={bg=windColor},
                         cell{11}{8}={bg=windOffColor},
                         cell{12}{8}={bg=otherResColor},
                         cell{13}{8}={bg=storageHydroColor},
                         cell{14}{8}={bg=pumpedStorageColor},
                         cell{15}{8}={bg=bessColor},
                         cell{16}{8}={bg=otherStorageColor},
                        }
                \textbf{Power Plant Type} & \textbf{German Name} & \textbf{LaTeX~Name} & \textbf{R} & \textbf{G} & \textbf{B} & \textbf{Hex} & \textbf{Color} \\
                     Oil & Öl & \texttt{\textbackslash oilColor} & 16 & 47 & 64 & 102F40 & \\
                     Coal & Kohle & \texttt{\textbackslash coalColor} & 127 & 127 & 127 & 7F7F7F &  \\
                     Gas & Gas & \texttt{\textbackslash gasColor} & 191 & 191 & 191 & BFBFBF &  \\
                     Other non-RES & Sonstige nicht-EE & \texttt{\textbackslash otherNonResColor} & 221 & 110 & 56 & DD6E38 &  \\
                     Nuclear & Nuklear & \texttt{\textbackslash nuclearColor} & 236 & 96 & 95 & EC605F &  \\
                     Biomass & Biomasse & \texttt{\textbackslash biomassColor} & 198 & 70 & 59 & C6463B &  \\
                     Run-of-River & Laufwasserkraft & \texttt{\textbackslash rorColor} & 0 & 176 & 240 & 00B0F0 &  \\
                     Solar/PV & Solar/PV & \texttt{\textbackslash pvColor} & 255 & 192 & 0 & FFC000 &  \\
                     Wind-onshore & Wind-onshore & \texttt{\textbackslash windColor} & 146 & 208 & 80 & 92D050 &  \\
                     Wind-offshore & Wind-offshore & \texttt{\textbackslash windOffColor} & 30 & 143 & 79 & 1E8F4F &  \\
                     Other RES & Sonstige EE & \texttt{\textbackslash otherResColor} & 185 & 207 & 222 & B9CFDE &  \\
                     Storage Hydro & Wasserspeicher & \texttt{\textbackslash storageHydroColor} & 75 & 172 & 198 & 4BACC6 &  \\
                     Pumped Storage & Pumpspeicher & \texttt{\textbackslash pumpedStorageColor} & 59 & 90 & 112 & 3B5A70 &  \\
                     BESS & Batteriespeicher & \texttt{\textbackslash bessColor} & 112 & 48 & 160 & 7030A0 &  \\
                     Other Storages & Sonstige Speicher & \texttt{\textbackslash otherStorageColor} & 76 & 133 & 123 & 4C857B &  \\
              \end{tblr}
              }
      \end{table}

\end{frame}

%% -------------------------------------------------------------------
%% Structure your Data with Clear Tables
\section{\textbf{Structure} your \textbf{Data} with Clear \textbf{Tables}}

\begin{frame}
    \agenda{Agenda}{} % No subtitle and highlights currentvsection
    %\agenda[sections=all]{Agenda}{Overview of lecture structure} % With subtitle and all sections highlighted
\end{frame}

\begin{frame}{Tables}
    \framesubtitle{Clear and Well-Designed Tables}

    \vspace{-0.7cm}
    \begin{minipage}[t]{0.49\textwidth}
        \begin{coloredblock}[turquoise][\centering\faIcon{table}~~~Table Design][\footnotesize]
            \vspace{0.2cm}
            \begin{tugitemize}
                \item \footnotesize \textbf{Bold headers} for clarity and emphasis
                \item \textbf{Black vertical lines} for \textbf{header} and \textbf{bottom} only (1.5~pt) 
                \item \textbf{Grey vertical lines} (1~pt) for the \textbf{rest}
                \item \footnotesize \textbf{Avoid horizontal lines} (unless necessary)
                \item \footnotesize \textbf{Use gray!20} as \textbf{background} for \textbf{odd rows}
            \end{tugitemize}
        \end{coloredblock}
        \begin{coloredblock}[yellow][\centering\faIcon{glasses}~~~Data Readability][\footnotesize]
            \vspace{0.2cm}
            \begin{tugitemize}
                \item \footnotesize \textbf{Align numbers by decimal separator} for better readability
                \item \footnotesize \textbf{Show only meaningful decimals} – no clutter!
                \item \footnotesize \textbf{Use consistent units} and place them in the header when possible
                \item \footnotesize \textbf{Highlight} the numbers you want to emphasizes
            \end{tugitemize}
        \end{coloredblock}
    \end{minipage}
    \hfill
    \begin{minipage}[t]{0.49\textwidth}
        \vspace{2cm}
        \begin{table}[htbp]
            \centering
            \small
            \begin{tblr}{
                  colspec = {
                        l 
                        S[table-format=4.1] % S for decimal-aligned numeric columns
                        S[table-format=3.1] % S for decimal-aligned numeric columns
                        S[table-format=3.1] % S for decimal-aligned numeric columns
                        S[table-format=3.1] % S for decimal-aligned numeric columns
                  },
                  hline{1-2}={solid,1.5pt}, % Header with black lines on top and bottom
                  hline{3-Y}={grey}, % Grey lines between rows
                  hline{Z}={solid,1.5pt}, % Black line at the end
                  row{odd}={grey!20}, % Background color odd rows
                  % row{even}={white}, % Background color even rows
                  row{1}={white,font=\bfseries}, % Bold font for header
                  cell{4}{2}={yellowLight, cmd=\textbf}, % Highlight cell (has to be after row{odd}={grey!20})
                  cell{4}{3}={yellowLight, cmd=\textbf}, % Highlight cell (has to be after row{odd}={grey!20})
                }
                \textbf{Power Plant Type} & \textbf{2022} & \textbf{2023} & \textbf{2024} & \textbf{2025} \\
                Coal           & 25.3 & 20.6 & 15.9 & 10.5 \\
                Gas            & 75.3 & 70.5 & 60.2 & 55.6 \\
                Run-of-River   & 40.3 & 45.0 & 50.0 & 50.0 \\
                PV             & 50.5 & 70.5 & 90.7 & 120.0 \\
                Wind           & 80.0 & 90.2 & 110.3 & 130.4 \\
                Storage Hydro  & 10.7 & 20.0 & 30.7 & 40.2 \\
            \end{tblr}
            \caption{\centering Installed capacity per power plant type in Wakanda from 2022--2025 in MW.}
        \end{table}
    \end{minipage}
    
\end{frame}


%% -------------------------------------------------------------------
%% Additional Tips to Improve your Design
\section{Additional \textbf{Tips} to Improve your Design}

\begin{frame}
    \agenda{Agenda}{} % No subtitle and highlights currentvsection
    %\agenda[sections=all]{Agenda}{Overview of lecture structure} % With subtitle and all sections highlighted
\end{frame}

\begin{frame}{Additional Tips}
    \framesubtitle{Improve your Design}

    \begin{coloredblock}[iee][\centering\faIcon{dog}~~~Icon]
        \begin{itemize}
            \item Incorporate \textbf{icons} to enhance \textbf{visual clarity and aid }quick \textbf{comprehension}
            \item The \textbf{recommended option} in \LaTeX~is to use icons from the \textbf{awesomefont package}. You can find all \textbf{available icons} in the \href{https://mirror.easyname.at/ctan/fonts/fontawesome5/doc/fontawesome5.pdf}{\textbf{docs}}
            \item When using \textbf{third-party icons}:
            \vspace{-0.5\topsep}
            \begin{itemize}
                \item Ensure the icons follow a \textbf{consistent design} style
                \item Verify that you have the appropriate \textbf{usage rights} for the icons
            \end{itemize}
        \end{itemize}
    \end{coloredblock}

        \begin{coloredblock}[yellow][\centering\faIcon{image}~~~Pictures]
            \begin{itemize}
                \item Always \textbf{credit} the \textbf{source} of your pictures
                \item Ensure all images are \textbf{crisp} and \textbf{sharp} for \textbf{optimal quality}
            \end{itemize}
        \end{coloredblock}
\end{frame}

\begin{frame}{Faded Image}
    \framesubtitle{Add a Nice Visual Touch}

    \insertfadedpicture{15.98cm}{figures/iee_besprechung.png}{Source: Institute of Electricity Economics and Energy Innovation/TU Graz}

    \vspace{-0.8cm}
    \begin{minipage}[t]{0.75\textwidth}
        \begin{coloredblock}[blue][\centering \faIcon{image}~~~Faded Image]
            \begin{itemize}
               \item To \textbf{add} a \textbf{faded image} as a background to your slide, use the \textbf{following command}:
                \item[] \begin{center}\footnotesize\texttt{\textbackslash insertfadedpicture\{inset from right\}\{image path\}\{Optional: caption text\}}\end{center}
                \begin{itemize}
                    \item \textbf{Inset from right}: Defines how far the image should be shifted in from the right edge. (Ensure the image is large enough.)
                    \item \textbf{Image path}: Specifies the file path to the image.
                    \item \textbf{Caption text}: Provides the text (e.g., image source) to be displayed in the bottom-right corner.
                \end{itemize}
                \item This \textbf{must be included at the beginning} of your \textbf{frame}.
            \end{itemize}     
        \end{coloredblock}
    \end{minipage}
\end{frame}


\begin{frame}{Lists}
    \framesubtitle{Different Styles of Lists}

    \begin{coloredblock}[yellow]
        \centering
        There are \textbf{different styles} of \textbf{lists} that you can choose from.
    \end{coloredblock}

    % Bottom blocks (side-by-side)
    \begin{minipage}[t]{0.49\textwidth}
        \begin{coloredblock}[blue][\centering\texttt{\textbf{\textbackslash begin\{itemize\}}}:][][3.7cm]
            \begin{itemize}
                \item Item level 1
                \begin{itemize}
                    \item Item level 2
                    \begin{itemize}
                        \item Item level 3
                    \end{itemize}
                \end{itemize}
            \end{itemize}
        \end{coloredblock}

        \begin{coloredblock}[iee][\centering\texttt{\textbf{\textbackslash begin\{enumerate\}}}:][][3.7cm]
            \begin{enumerate}
            \item Item level 1
                \begin{enumerate}
                    \item Item level 2
                    \begin{enumerate}
                        \item Item level 3
                    \end{enumerate}
                \end{enumerate}
            \end{enumerate}
        \end{coloredblock}
        
    \end{minipage}
    \hfill
    \begin{minipage}[t]{0.49\textwidth}
        \begin{coloredblock}[green][\centering\texttt{\textbackslash begin\{tugitemize\}} (tighter spacing):][][3.7cm]
            \vspace{0.5cm}
            \begin{tugitemize}
                \item Item level 1
                \begin{tugitemize}
                    \item Item level 2
                    \begin{tugitemize}
                        \item Item level 3
                    \end{tugitemize}
                \end{tugitemize}
            \end{tugitemize}
        \end{coloredblock}

        \begin{coloredblock}[grey][\centering\texttt{\textbackslash begin\{boxenumerate\}}:][][3.7cm]
                \begin{boxenumerate}
                    \item Item level 1
                    \begin{boxenumerate}
                        \item Item level 2
                        \begin{boxenumerate}
                            \item Item level 3
                        \end{boxenumerate}
                    \end{boxenumerate}
                \end{boxenumerate}
        \end{coloredblock}
    \end{minipage}
\end{frame}


%% -------------------------------------------------------------------
%% Visualize your Numbers with Plots
\section{\textbf{Acknowledge} Work with Proper \textbf{Citations}}

\begin{frame}
    \agenda{Agenda}{} % No subtitle and highlights currentvsection
    %\agenda[sections=all]{Agenda}{Overview of lecture structure} % With subtitle and all sections highlighted
\end{frame}


\begin{frame}{References}
    \framesubtitle{Cite the Sources you use}

  \begin{coloredblock}[grey]
    \centering
      ”\textit{In the middle of every difficulty lies opportunity.}“
      
      \vspace{0.7cm}
      \scriptsize \textcite{Einstein2018}
  \end{coloredblock}

  \begin{coloredblock}[blue][\centering\texttt{\textbackslash textcite\{\}}]
      If you want to include the authors, use \texttt{\textbackslash textcite\{\}}.

      \vspace{0.5cm}
      In their study, \textbf{\textcite{gaugl2023}} demonstrate that rising CO2 prices result in higher electricity prices in Austria.
  \end{coloredblock}

  \begin{coloredblock}[yellow][\centering\texttt{\textbackslash cite\{\}}]
      A simple reference is accomplished with \texttt{\textbackslash cite\{\}}.

      \vspace{0.5cm}
      The LEGO model is an energy system optimization model developed at the Institute of Electricity Economics and Energy Innovation. \textbf{\cite{wogrin2022}}
  \end{coloredblock}

\end{frame}

%% -------------------------------------------------------------------
%% Visualize your Numbers with Plots
\section{Explore \textbf{Design Inspirations} to Spark  your Creativity}

\sectionheader[Colorful \& Monocolor]{Design Inspirations}

\begin{frame}{Methodology}
    \framesubtitle{NTC Model}
 
    \begin{itemize}
        \item \textbf{Objective Function}: Minimize total system costs
        \vspace{-0.3cm}
        \item[] \begin{center}
                \footnotesize$\min \sum_{g,h} c_{g}^{op} \cdot p_g^{h} + \sum_{g} x_g \cdot (c_{g}^{Inv,MW} + c_{g}^{Inv,MWh} \cdot E2P_g)$
                \end{center}
        \vspace{-0.15cm}
        \item \textbf{Constraint 1}: Balance equation
        \vspace{-0.3cm}
        \item[] \begin{center}
                \footnotesize$\sum_{g} p_{gz(g,z),h} - \sum_{g} p_{cs(g,z),h} - \sum_{z \neq y} exp_{z,y,h} + \sum_{z \neq y} imp{z,y,h} = dem_{z,h}$
                \end{center}
        \vspace{-0.15cm}
        \item \textbf{Constraint 2}: NTC limits
        \vspace{-0.3cm}
        \item[] \begin{center}
                \footnotesize$exp_{z,y,h} \leq NTC_{z,y} \cdot bn_{z,y,h}$
                \end{center}
        \vspace{-0.15cm}
        \item \textbf{Constraint 3}: Flow direction
        \vspace{-0.3cm}
        \item[] \begin{center}
                \footnotesize$imp_{z,y,h} \leq NTC_{y,z} \cdot (1-bn_{z,y,h})$
                \end{center}
        \vspace{-0.15cm}
        \item \textbf{Constraint 4}: Export/Import
        \vspace{-0.3cm}
        \item[] \begin{center}
                \footnotesize$exp_{z,y,h} = imp_{y,z,h}$
                \end{center}
        \vspace{-0.15cm}
        \item \textbf{Constraint 5}: Generator Limits
        \vspace{-0.3cm}
        \item[] \begin{center}
                \footnotesize$\underline{p_g} \leq p_{g,h} \leq \overline{p_g}$
                \end{center}
    \end{itemize}

\end{frame}

\begin{frame}{Methodology}
    \framesubtitle{NTC Model}

        \vspace{-0.5cm}
    \begin{minipage}[t]{0.2\textwidth}
        \begin{coloredblockdark}[blue][][][1.5cm]
            \makebox[\textwidth]{%
                \begin{minipage}[c][1.5cm][c]{0.3\textwidth}
                    \centering
                    \Large{\faIcon{coins}}
                \end{minipage}%
                \hfill
                \begin{minipage}[c][1.5cm][c]{0.69\textwidth}
                    \centering
                    \tiny\textbf{Objective Function:}\\
                    Minimize total system costs
                \end{minipage}%
            }
        \end{coloredblockdark}

        \begin{coloredblockdark}[blue][][][1.5cm]
            \makebox[\textwidth]{%
                \begin{minipage}[c][1.5cm][c]{0.3\textwidth}
                    \centering
                    \Large{\faIcon{balance-scale}}
                \end{minipage}%
                \hfill
                \begin{minipage}[c][1.5cm][c]{0.69\textwidth}
                    \centering
                    \tiny\textbf{Constraint 1:}\\ 
                    Balance equation
                \end{minipage}%
            }
        \end{coloredblockdark}

        \begin{coloredblockdark}[blue][][][1.5cm]
            \makebox[\textwidth]{%
                \begin{minipage}[c][1.5cm][c]{0.3\textwidth}
                    \centering
                    \Large{\faIcon{hand-paper}}
                \end{minipage}%
                \hfill
                \begin{minipage}[c][1.5cm][c]{0.69\textwidth}
                    \centering
                    \tiny\textbf{Constraint 2:}\\ 
                    NTC Limits
                \end{minipage}%
            }
        \end{coloredblockdark}

        \begin{coloredblockdark}[blue][][][1.5cm]
            \makebox[\textwidth]{%
                \begin{minipage}[c][1.5cm][c]{0.3\textwidth}
                    \centering
                    \Large{\faIcon{arrows-alt-v}}
                \end{minipage}%
                \hfill
                \begin{minipage}[c][1.5cm][c]{0.69\textwidth}
                    \centering
                    \tiny\textbf{Constraint 3:}\\ 
                    Flow Direction
                \end{minipage}%
            }
        \end{coloredblockdark}

        \begin{coloredblockdark}[blue][][][1.5cm]
            \makebox[\textwidth]{%
                \begin{minipage}[c][1.5cm][c]{0.3\textwidth}
                    \centering
                    \Large{\faIcon{arrows-alt-h}}
                \end{minipage}%
                \hfill
                \begin{minipage}[c][1.5cm][c]{0.69\textwidth}
                    \centering
                    \tiny\textbf{Constraint 4:}\\ 
                    Export/Import
                \end{minipage}%
            }
        \end{coloredblockdark}

        \begin{coloredblockdark}[blue][][][1.5cm]
            \makebox[\textwidth]{%
                \begin{minipage}[c][1.5cm][c]{0.3\textwidth}
                    \centering
                    \Large{\faIcon{power-off}}
                \end{minipage}%
                \hfill
                \begin{minipage}[c][1.5cm][c]{0.69\textwidth}
                    \centering
                    \tiny\textbf{Constraint 5:}\\ 
                    Generator Limits
                \end{minipage}%
            }
        \end{coloredblockdark}
    \end{minipage}
    \hfill
    \begin{minipage}[t]{0.78\textwidth}
        \begin{coloredblock}[blue][][][1.5cm]
            \begin{center}
                \small$\min \sum_{g,h} c_{g}^{op} \cdot p_g^{h} + \sum_{g} x_g \cdot (c_{g}^{Inv,MW} + c_{g}^{Inv,MWh} \cdot E2P_g)$
            \end{center}
        \end{coloredblock}

        \begin{coloredblock}[blue][][][1.5cm]
            \begin{center}
                \small$\sum_{g} p_{gz(g,z),h} - \sum_{g} p_{cs(g,z),h} - \sum_{z \neq y} exp_{z,y,h} + \sum_{z \neq y} imp{z,y,h} = dem_{z,h}$
            \end{center}
        \end{coloredblock}

        \begin{coloredblock}[blue][][][1.5cm]
            \begin{center}
                \small$exp_{z,y,h} \leq NTC_{z,y} \cdot bn_{z,y,h}$
            \end{center}
        \end{coloredblock}

        \begin{coloredblock}[blue][][][1.5cm]
            \begin{center}
                \small$imp_{z,y,h} \leq NTC_{y,z} \cdot (1-bn_{z,y,h})$
            \end{center}
        \end{coloredblock}

        \begin{coloredblock}[blue][][][1.5cm]
            \begin{center}
                \small$exp_{z,y,h} = imp_{y,z,h}$
            \end{center}
        \end{coloredblock}

        \begin{coloredblock}[blue][][][1.5cm]
            \begin{center}
                \small$\underline{p_g} \leq p_{g,h} \leq \overline{p_g}$
            \end{center}
        \end{coloredblock}
    \end{minipage}
    
\end{frame}

\begin{frame}{Examined Main Scenarios}
    \framesubtitle{Analysis of Flexibility Requirements}

    \begin{tugitemize}
        \item \small\textbf{Base Scenario}
        \vspace{-0.3cm}
        \begin{tugitemize}
            \item \footnotesize STORYLINE: Austria pursues a \textbf{highly ambitious path} to \textbf{decarbonization} and achieves the goal of decarbonization \textbf{by 2040}:
            \begin{tugitemize}
                \item \scriptsize Ambitious renewable energy expansion \textbf{aligned with the ÖNIP} (Austrian National Energy and Climate Plan).
            \end{tugitemize}
        \end{tugitemize}
        \item \small\textbf{„Reduced Electricity Transport“ RET}
        \vspace{-0.3cm}
        \begin{tugitemize}
            \item \footnotesize STORYLINE: \textbf{Electricity transmission capacities} are \textbf{lower} than assumed in the base scenario:
            \begin{tugitemize}
                \item \scriptsize\textbf{Grid expansion is delayed}
            \end{tugitemize}
            \item \footnotesize The analysis examines how the reduced transmission capacities in the transmission network, compared to the baseline scenario, impact controllable electricity generation capacities, storage requirements, additional costs, etc.
        \end{tugitemize}
        \item \small\textbf{„Extreme Weather Years“ EWY}
        \vspace{-0.3cm}
        \begin{tugitemize}
            \item \footnotesize STORYLINE: To obtain reliable results, \textbf{variations in the assumed weather data} are \textbf{considered}:
            \begin{tugitemize}
                \item \scriptsize An \textbf{extreme event} is simulated, focusing on a winter "\textbf{Dunkelflaute}"
            \end{tugitemize}
            \item \footnotesize The analysis examines how the extreme weather event impacts the generation fleet, storage requirements, transmission corridors, and the overall systemic behavior.
        \end{tugitemize}
        \centering
        \textbf{Through the in-depth analysis of the baseline scenario  and the comparison with the extreme scenarios, the research questions are answered.}
    \end{tugitemize}
    
\end{frame}

\begin{frame}{Examined Main Scenarios}
    \framesubtitle{Analysis of Flexibility Requirements}

    \vspace{-0.8cm}
    \begin{minipage}[t][12.9cm]{\textwidth}
        \begin{minipage}[t]{0.325\textwidth}
            \begin{coloredblock}[green][\centering „Base: Net Zero 2040“\\Base][\footnotesize]
                \begin{minipage}[t][3cm]{0.9\textwidth} 
                    \scriptsize STORYLINE: Austria pursues a \textbf{highly ambitious path} to \textbf{decarbonization} and achieves the goal of decarbonization \textbf{by 2040}:
                \end{minipage}
                \begin{minipage}[t][2.5cm]{0.9\textwidth}
                    \begin{itemize}
                        \item \scriptsize Ambitious renewable energy expansion \textbf{aligned with the ÖNIP} (Austrian National Energy and Climate Plan).
                    \end{itemize}
                \end{minipage}
                \begin{minipage}[t][3.7cm]{0.9\textwidth} 
                    \scriptsize ~
                \end{minipage}
            \end{coloredblock}
        \end{minipage}
        \hfill
        \begin{minipage}[t]{0.325\textwidth}
            \begin{coloredblock}[yellow][\centering „Reduced Electricity Transport“\\RET][\footnotesize]
                \begin{minipage}[t][3cm]{0.9\textwidth} 
                    \scriptsize STORYLINE: \textbf{Electricity transmission capacities} are \textbf{lower} than assumed in the base scenario:
                \end{minipage}
                \begin{minipage}[t][2.5cm]{0.9\textwidth}
                    \begin{itemize}
                        \item \scriptsize \textbf{Grid expansion is delayed}
                    \end{itemize}
                \end{minipage}
                \begin{minipage}[t][3.7cm]{0.9\textwidth} 
                    \scriptsize \textit{The analysis examines how the reduced transmission capacities in the transmission network, compared to the baseline scenario, impact controllable electricity generation capacities, storage requirements, additional costs, etc.}
                \end{minipage}
            \end{coloredblock}
        \end{minipage}
        \hfill
        \begin{minipage}[t]{0.325\textwidth}
            \begin{coloredblock}[blue][\centering „Extreme Weather Years“\\ EWY][\footnotesize]
                \begin{minipage}[t][3cm]{0.9\textwidth} 
                    \scriptsize STORYLINE: To obtain reliable results, \textbf{variations in the assumed weather data} are \textbf{considered}:
                \end{minipage}
                \begin{minipage}[t][2.5cm]{0.9\textwidth}
                    \begin{itemize}
                        \item \scriptsize An \textbf{extreme event} is simulated, focusing on a winter "\textbf{Dunkelflaute}"
                    \end{itemize}
                \end{minipage}
                \begin{minipage}[t][3.7cm]{0.9\textwidth} 
                    \scriptsize \textit{The analysis examines how the extreme weather event impacts the generation fleet, storage requirements, transmission corridors, and the overall systemic behavior.}
                \end{minipage}
            \end{coloredblock}
        \end{minipage}
        
    \end{minipage}

    \begin{coloredblock}[grey]
        \centering
        \footnotesize\textbf{Through the in-depth analysis of the baseline scenario \\
        and the comparison with the extreme scenarios, the research questions are answered.}
    \end{coloredblock}

\end{frame}

\begin{frame}{Examined Main Scenarios}
    \framesubtitle{Analysis of Flexibility Requirements}

    \vspace{-0.8cm}
    \begin{minipage}[t][12.9cm]{\textwidth}
        \begin{minipage}[t]{0.325\textwidth}
            \begin{coloredblock}[blue][\centering „Base: Net Zero 2040“\\Base][\footnotesize]
                \begin{minipage}[t][3cm]{0.9\textwidth} 
                    \scriptsize STORYLINE: Austria pursues a \textbf{highly ambitious path} to \textbf{decarbonization} and achieves the goal of decarbonization \textbf{by 2040}:
                \end{minipage}
                \begin{minipage}[t][2.5cm]{0.9\textwidth}
                    \begin{itemize}
                        \item \scriptsize Ambitious renewable energy expansion \textbf{aligned with the ÖNIP} (Austrian National Energy and Climate Plan).
                    \end{itemize}
                \end{minipage}
                \begin{minipage}[t][3.7cm]{0.9\textwidth} 
                    \scriptsize ~
                \end{minipage}
            \end{coloredblock}
        \end{minipage}
        \hfill
        \begin{minipage}[t]{0.325\textwidth}
            \begin{coloredblock}[blue][\centering „Reduced Electricity Transport“\\RET][\footnotesize]
                \begin{minipage}[t][3cm]{0.9\textwidth} 
                    \scriptsize STORYLINE: \textbf{Electricity transmission capacities} are \textbf{lower} than assumed in the base scenario:
                \end{minipage}
                \begin{minipage}[t][2.5cm]{0.9\textwidth}
                    \begin{itemize}
                        \item \scriptsize \textbf{Grid expansion is delayed}
                    \end{itemize}
                \end{minipage}
                \begin{minipage}[t][3.7cm]{0.9\textwidth} 
                    \scriptsize \textit{The analysis examines how the reduced transmission capacities in the transmission network, compared to the baseline scenario, impact controllable electricity generation capacities, storage requirements, additional costs, etc.}
                \end{minipage}
            \end{coloredblock}
        \end{minipage}
        \hfill
        \begin{minipage}[t]{0.325\textwidth}
            \begin{coloredblock}[blue][\centering „Extreme Weather Years“\\ EWY][\footnotesize]
                \begin{minipage}[t][3cm]{0.9\textwidth} 
                    \scriptsize STORYLINE: To obtain reliable results, \textbf{variations in the assumed weather data} are \textbf{considered}:
                \end{minipage}
                \begin{minipage}[t][2.5cm]{0.9\textwidth}
                    \begin{itemize}
                        \item \scriptsize An \textbf{extreme event} is simulated, focusing on a winter "\textbf{Dunkelflaute}"
                    \end{itemize}
                \end{minipage}
                \begin{minipage}[t][3.7cm]{0.9\textwidth} 
                    \scriptsize \textit{The analysis examines how the extreme weather event impacts the generation fleet, storage requirements, transmission corridors, and the overall systemic behavior.}
                \end{minipage}
            \end{coloredblock}
        \end{minipage}
        
    \end{minipage}

    \begin{coloredblock}[grey]
        \centering
        \footnotesize\textbf{Through the in-depth analysis of the baseline scenario \\
        and the comparison with the extreme scenarios, the research questions are answered.}
    \end{coloredblock}

\end{frame}


\begin{frame}{Grading}
    \vspace{-0.5cm}
        \begin{minipage}[t]{0.49\textwidth}
            \begin{tugitemize}
                \item Attendance mandatory
                \begin{tugitemize}
                    \item If more than two lessons are missed without an adequate excuse, you can no longer successfully complete the course.
                \end{tugitemize}
                \item Participation (50 Points)
                \begin{tugitemize}
                    \item Preparatory tasks\\
                    Short tasks before class like watching a video, reading a text and answering some questions in TeachCenter
                    \item Exercises during class
                \end{tugitemize}
                \item Homework 1 \& 2 (25 Points each)
                \begin{tugitemize}
                    \item Exercise\\
                    Work you have to do at home.
                    \item Report\\
                    Comment your code and write a small report
                \end{tugitemize}
            \end{tugitemize}
        \end{minipage}
        \hfill
        \begin{minipage}[t]{0.49\textwidth}
            \begin{tugitemize}
                \item In each part more than 50\% of the points are required for a positive grade
                \item Grading scale
                \begin{tugitemize}
                    \item \textless50 Points: 5
                    \item 50 – 62 Points: 4
                    \item 63 – 75 Points: 3
                    \item 76 – 88 Points: 2
                    \item 89 – 100 Points: 1
                \end{tugitemize}
            \end{tugitemize}
        \end{minipage}
    
\end{frame}


\begin{frame}{Grading}
    \begin{coloredblock}[turquoise]
        \centering\footnotesize\textbf{Attendance mandatory\textsuperscript{1}}
    \end{coloredblock}

    \vspace{-0.5cm}
    \begin{minipage}[t][9.5cm]{\textwidth}
        \begin{minipage}[t]{0.32\textwidth}
            \begin{coloredblock}[blue][\centering Participation][\footnotesize][5.5cm]
                \begin{itemize}
                    \item \footnotesize \textbf{Preparatory tasks}\\
                    Short tasks before class like watching a video, reading a text and answering some questions in TeachCenter
                    \item \footnotesize \textbf{Exercises during class}
                \end{itemize}
            \end{coloredblock}
            \centering \footnotesize \textbf{50 Point}
        \end{minipage}
        \hfill
        \begin{minipage}[t]{0.32\textwidth}
            \begin{coloredblock}[blue][\centering Homework 1][\footnotesize][5.5cm]
                \begin{itemize}
                    \item \footnotesize \textbf{Exercise}\\
                    Work you have to do at home.
                    \item \footnotesize \textbf{Report}\\
                    Comment your code and write a small report
                \end{itemize}
            \end{coloredblock}
            \centering \footnotesize \textbf{25 Point}
        \end{minipage}
        \hfill
        \begin{minipage}[t]{0.32\textwidth}
            \begin{coloredblock}[blue][\centering Homework 2][\footnotesize][5.5cm]
                \begin{itemize}
                    \item \footnotesize \textbf{Exercise}\\
                    Work you have to do at home.
                    \item \footnotesize \textbf{Report}\\
                    Comment your code and write a small report
                \end{itemize}
            \end{coloredblock}
            \centering \footnotesize \textbf{25 Point}
        \end{minipage}
    \end{minipage}
    
    \begin{coloredblock}[yellow]
        \centering\footnotesize\textbf{In each part more than 50\% of the points are required for a positive grade}
    \end{coloredblock}

    \vspace{0.5cm}
    \begin{coloredblock}[green]
        \begin{minipage}[c]{0.24\textwidth}
            \centering\footnotesize \textbf{Grading Scale}
        \end{minipage}
        \hfill
        \begin{minipage}[c]{0.24\textwidth}
            \footnotesize
            <50 Points:         5\\
            50 – 62 Points:     4
        \end{minipage}
        \hfill
        \begin{minipage}[c]{0.24\textwidth}
            \footnotesize
            63 – 75 Points:     3\\
            76 – 88 Points:     2
        \end{minipage}
        \hfill
        \begin{minipage}[t]{0.24\textwidth}
            \footnotesize
            89 – 100 Points:    1
        \end{minipage}
        
    \end{coloredblock}

    \begin{tikzpicture}[remember picture, overlay]
     \node[anchor=south east, xshift=-0.5cm, yshift=0.6cm] at (current page.south east) {
       {\myFootnotesize \textsuperscript{1} If more than two lessons are missed without an adequate excuse, you can no longer successfully complete the course.}
     };
   \end{tikzpicture}
\end{frame}


\begin{frame}{Grading}
    \begin{coloredblock}[blue]
        \centering\footnotesize\textbf{Attendance mandatory\textsuperscript{1}}
    \end{coloredblock}

    \vspace{-0.5cm}
    \begin{minipage}[t][9.5cm]{\textwidth}
        \begin{minipage}[t]{0.32\textwidth}
            \begin{coloredblock}[blue][\centering Participation][\footnotesize][5.5cm]
                \begin{itemize}
                    \item \footnotesize \textbf{Preparatory tasks}\\
                    Short tasks before class like watching a video, reading a text and answering some questions in TeachCenter
                    \item \footnotesize \textbf{Exercises during class}
                \end{itemize}
            \end{coloredblock}
            \centering \footnotesize \textbf{50 Point}
        \end{minipage}
        \hfill
        \begin{minipage}[t]{0.32\textwidth}
            \begin{coloredblock}[blue][\centering Homework 1][\footnotesize][5.5cm]
                \begin{itemize}
                    \item \footnotesize \textbf{Exercise}\\
                    Work you have to do at home.
                    \item \footnotesize \textbf{Report}\\
                    Comment your code and write a small report
                \end{itemize}
            \end{coloredblock}
            \centering \footnotesize \textbf{25 Point}
        \end{minipage}
        \hfill
        \begin{minipage}[t]{0.32\textwidth}
            \begin{coloredblock}[blue][\centering Homework 2][\footnotesize][5.5cm]
                \begin{itemize}
                    \item \footnotesize \textbf{Exercise}\\
                    Work you have to do at home.
                    \item \footnotesize \textbf{Report}\\
                    Comment your code and write a small report
                \end{itemize}
            \end{coloredblock}
            \centering \footnotesize \textbf{25 Point}
        \end{minipage}
    \end{minipage}
    
    \begin{coloredblock}[blue]
        \centering\footnotesize\textbf{In each part more than 50\% of the points are required for a positive grade}
    \end{coloredblock}

    \vspace{0.5cm}
    \begin{coloredblock}[blue]
        \begin{minipage}[c]{0.24\textwidth}
            \centering\footnotesize \textbf{Grading Scale}
        \end{minipage}
        \hfill
        \begin{minipage}[c]{0.24\textwidth}
            \footnotesize
            <50 Points:         5\\
            50 – 62 Points:     4
        \end{minipage}
        \hfill
        \begin{minipage}[c]{0.24\textwidth}
            \footnotesize
            63 – 75 Points:     3\\
            76 – 88 Points:     2
        \end{minipage}
        \hfill
        \begin{minipage}[t]{0.24\textwidth}
            \footnotesize
            89 – 100 Points:    1
        \end{minipage}
        
    \end{coloredblock}

    \begin{tikzpicture}[remember picture, overlay]
     \node[anchor=south east, xshift=-0.5cm, yshift=0.6cm] at (current page.south east) {
       {\myFootnotesize \textsuperscript{1} If more than two lessons are missed without an adequate excuse, you can no longer successfully complete the course.}
     };
   \end{tikzpicture}
\end{frame}

\begin{frame}{Classification}
    \framesubtitle{Definition}
    \begin{tugitemize}
        \item \small\textbf{Classification} is a \textbf{machine learning technique} used to \textbf{predict group membership} for data instances.
        \item \small Training Data
        \begin{tugitemize}
            \item \scriptsize Given a collection of records (training set), each record contains a set of attributes, with one attribute being the class.
            \item \scriptsize Each sample is characterized by a tuple (X, y), where:
            \item \scriptsize X: attribute set (predictor, independent variable, input)
            \item \scriptsize y: class label (response, dependent variable, output)
        \end{tugitemize}
        \item \small Task
        \begin{tugitemize}
            \item \scriptsize Learn a model that maps each attribute set X to one of the predefined class labels y.
            \item \scriptsize Find a model for the class attribute as a function of the values of other attributes.
        \end{tugitemize}
        \item \small Goal
        \begin{tugitemize}
            \item \scriptsize Assign previously unseen records to a class as accurately as possible.
            \item \scriptsize Use a test set to determine the accuracy of the model.
            \item \scriptsize Typically, the dataset is divided into training and test sets:
            \item \scriptsize Training set: Used to build the model
            \item \scriptsize Test set: Used to validate the model
        \end{tugitemize}
    \end{tugitemize}
\end{frame}

\begin{frame}{Classification}
    \framesubtitle{Definition}
    \small\centering\textbf{Classification} is a \textbf{machine learning technique} used to \textbf{predict group membership} for data instances.

    \vspace{0.2cm}
    \begin{coloredblock}[yellow][\faIcon{table}~~~Training Data][\footnotesize]
        \vspace{0.2cm}
        \begin{tugitemize}
            \item \scriptsize Given a collection of records (training set), each record contains a set of attributes, with one attribute being the class.
            \item \scriptsize Each sample is characterized by a tuple (X, y), where:
            \item \scriptsize X: attribute set (predictor, independent variable, input)
            \item \scriptsize y: class label (response, dependent variable, output)
        \end{tugitemize}
    \end{coloredblock}

    \begin{coloredblock}[green][\faIcon{tasks}~~~Task][\footnotesize]
        \vspace{0.2cm}
        \begin{tugitemize}
            \item \scriptsize Learn a model that maps each attribute set X to one of the predefined class labels y.
            \item \scriptsize Find a model for the class attribute as a function of the values of other attributes.
        \end{tugitemize}
    \end{coloredblock}

    \begin{coloredblock}[blue][\faIcon{bullseye}~~~Goal][\footnotesize]
        \vspace{0.2cm}
        \begin{tugitemize}
            \item \scriptsize Assign previously unseen records to a class as accurately as possible.
            \item \scriptsize Use a test set to determine the accuracy of the model.
            \item \scriptsize Typically, the dataset is divided into training and test sets:
            \item \scriptsize Training set: Used to build the model
            \item \scriptsize Test set: Used to validate the model
        \end{tugitemize}
    \end{coloredblock}

\end{frame}

\begin{frame}{Classification}
    \framesubtitle{Definition}
    \small\centering\textbf{Classification} is a \textbf{machine learning technique} used to \textbf{predict group membership} for data instances.

    \vspace{0.2cm}
    \begin{coloredblock}[blue][\faIcon{table}~~~Training Data][\footnotesize]
        \vspace{0.2cm}
        \begin{tugitemize}
            \item \scriptsize Given a collection of records (training set), each record contains a set of attributes, with one attribute being the class.
            \item \scriptsize Each sample is characterized by a tuple (X, y), where:
            \item \scriptsize X: attribute set (predictor, independent variable, input)
            \item \scriptsize y: class label (response, dependent variable, output)
        \end{tugitemize}
    \end{coloredblock}

    \begin{coloredblock}[blue][\faIcon{tasks}~~~Task][\footnotesize]
        \vspace{0.2cm}
        \begin{tugitemize}
            \item \scriptsize Learn a model that maps each attribute set X to one of the predefined class labels y.
            \item \scriptsize Find a model for the class attribute as a function of the values of other attributes.
        \end{tugitemize}
    \end{coloredblock}

    \begin{coloredblock}[blue][\faIcon{bullseye}~~~Goal][\footnotesize]
        \vspace{0.2cm}
        \begin{tugitemize}
            \item \scriptsize Assign previously unseen records to a class as accurately as possible.
            \item \scriptsize Use a test set to determine the accuracy of the model.
            \item \scriptsize Typically, the dataset is divided into training and test sets:
            \item \scriptsize Training set: Used to build the model
            \item \scriptsize Test set: Used to validate the model
        \end{tugitemize}
    \end{coloredblock}

\end{frame}


\begin{frame}{Classification}
    \framesubtitle{Examples}

    \begin{minipage}[t]{.49\textwidth}
        \begin{tugitemize}
            \item \small Sentiment Analysis
            \begin{tugitemize}
                \item \scriptsize Task: Determine the sentiment of a text (e.g., reviews, social media posts).
                \item \scriptsize Attributes: Text data (words, phrases).
                \item \scriptsize Class Labels: "Positive," "Negative," or "Neutral“
            \end{tugitemize}
            \item \small Disease Diagnosis
            \begin{tugitemize}
                \item \scriptsize Task: Diagnose whether a patient has a particular disease.
                \item \scriptsize Attributes: Medical history, test results, symptoms.
                \item \scriptsize Class Labels: "Disease" or "No Disease“
            \end{tugitemize}
            \item \small Activity Recognition
            \begin{tugitemize}
                \item \scriptsize Task: Determine the activity being performed based on sensor data.
                \item \scriptsize Attributes: Sensor readings (e.g., from a smartphone or wearable device).
                \item \scriptsize Class Labels: “Biking," "Running," etc.
            \end{tugitemize}
        \end{tugitemize}
    \end{minipage}
    \hfill
    \begin{minipage}[t]{.49\textwidth}
        \begin{tugitemize}
            \item \small Image Recognition
            \begin{tugitemize}
                \item \scriptsize Task: Identify objects or people in images.
                \item \scriptsize Attributes: Pixel values, image features.
                \item \scriptsize Class Labels: "Cat," "Dog," etc.
            \end{tugitemize}
            \item \small Language Detection
            \begin{tugitemize}
                \item \scriptsize Task: Identify the language of a given text.
                \item \scriptsize Attributes: Text data (words, characters).
                \item \scriptsize Class Labels: "English," "Spanish," etc.
            \end{tugitemize}
            \item \small Email Spam Detection
            \begin{tugitemize}
                \item \scriptsize Task: Classify as "spam" or "not spam."
                \item \scriptsize Attributes: Email content, sender, etc.
                \item \scriptsize Class Labels: "Spam" or "Not Spam"
            \end{tugitemize}
        \end{tugitemize}
    \end{minipage}

\end{frame}


\begin{frame}{Classification}
    \framesubtitle{Examples}

    \vspace{-0.8cm}
    \begin{minipage}[t]{0.32\textwidth}
    
        \begin{coloredblock}[yellow][\centering\faIcon[regular]{smile-beam}~~~Sentiment Analysis][\footnotesize][4.8cm]
            \begin{itemize}
                \item \scriptsize Task: Determine the sentiment of a text (e.g., reviews, social media posts).
                \item \scriptsize Attributes: Text data (words, phrases).
                \item \scriptsize Class Labels: "Positive," "Negative," or "Neutral"
            \end{itemize}
        \end{coloredblock}

        \begin{coloredblock}[red][\centering\faIcon{image}~~~Image Recognition][\footnotesize][4.8cm]
                \begin{itemize}
                    \item \scriptsize Task: Identify objects or people in images.
                    \item \scriptsize Attributes: Pixel values, image features, etc.
                    \item \scriptsize Class Labels: "Cat," "Dog," etc.
                \end{itemize}
        \end{coloredblock}
        
    \end{minipage}
    \hfill
    \begin{minipage}[t]{0.32\textwidth}
        \begin{coloredblock}[blue][\centering\faIcon{stethoscope}~~~Disease Diagnosis][\footnotesize][4.8cm]
            \begin{itemize}
                \item \scriptsize Task: Diagnose whether a patient has a particular disease.
                \item \scriptsize Attributes: Medical history, test results, symptoms.
                \item \scriptsize Class Labels: "Disease" / "No Disease"
            \end{itemize}
        \end{coloredblock}

        \begin{coloredblock}[turquoise][\centering\faIcon{comments}~~~Language Detection][\footnotesize][4.8cm]
            \begin{itemize}
                \item \scriptsize Task: Identify the language of a given text.
                \item \scriptsize Attributes: Text data (words, characters).
                \item \scriptsize Class Labels: "English," "Spanish," etc.
            \end{itemize}
        \end{coloredblock}
        
    \end{minipage}
    \hfill
    \begin{minipage}[t]{0.32\textwidth}

        \begin{coloredblock}[green][\centering\faIcon{running}~~~Activity Recognition][\footnotesize][4.8cm]
            \begin{itemize}
                \item \scriptsize Task: Determine the activity being performed based on sensor data.
                \item \scriptsize Attributes: Sensor readings (e.g., from a smartphone or wearable device).
                \item \scriptsize Class Labels: “Biking," "Running," etc.
            \end{itemize}
        \end{coloredblock}

        \begin{coloredblock}[grey][\centering\faIcon{envelope}~~~E-Mail Spam Detection][\footnotesize][4.8cm]
            \begin{itemize}
                \item \scriptsize Task: Classify as "spam" or "not spam."
                \item \scriptsize Attributes: Email content, sender, attachements etc.
                \item \scriptsize Class Labels: "Spam" or "Not Spam"
            \end{itemize}
        \end{coloredblock}
        
    \end{minipage}
\end{frame}


\begin{frame}{Classification}
    \framesubtitle{Examples}

    \vspace{-0.8cm}
    \begin{minipage}[t]{0.32\textwidth}
    
        \begin{coloredblock}[blue][\centering\faIcon[regular]{smile-beam}~~~Sentiment Analysis][\footnotesize][4.8cm]
            \begin{itemize}
                \item \scriptsize Task: Determine the sentiment of a text (e.g., reviews, social media posts).
                \item \scriptsize Attributes: Text data (words, phrases).
                \item \scriptsize Class Labels: "Positive," "Negative," or "Neutral"
            \end{itemize}
        \end{coloredblock}

        \begin{coloredblock}[blue][\centering\faIcon{image}~~~Image Recognition][\footnotesize][4.8cm]
                \begin{itemize}
                    \item \scriptsize Task: Identify objects or people in images.
                    \item \scriptsize Attributes: Pixel values, image features, etc.
                    \item \scriptsize Class Labels: "Cat," "Dog," etc.
                \end{itemize}
        \end{coloredblock}
        
    \end{minipage}
    \hfill
    \begin{minipage}[t]{0.32\textwidth}
        \begin{coloredblock}[blue][\centering\faIcon{stethoscope}~~~Disease Diagnosis][\footnotesize][4.8cm]
            \begin{itemize}
                \item \scriptsize Task: Diagnose whether a patient has a particular disease.
                \item \scriptsize Attributes: Medical history, test results, symptoms.
                \item \scriptsize Class Labels: "Disease" / "No Disease"
            \end{itemize}
        \end{coloredblock}

        \begin{coloredblock}[blue][\centering\faIcon{comments}~~~Language Detection][\footnotesize][4.8cm]
            \begin{itemize}
                \item \scriptsize Task: Identify the language of a given text.
                \item \scriptsize Attributes: Text data (words, characters).
                \item \scriptsize Class Labels: "English," "Spanish," etc.
            \end{itemize}
        \end{coloredblock}
        
    \end{minipage}
    \hfill
    \begin{minipage}[t]{0.32\textwidth}

        \begin{coloredblock}[blue][\centering\faIcon{running}~~~Activity Recognition][\footnotesize][4.8cm]
            \begin{itemize}
                \item \scriptsize Task: Determine the activity being performed based on sensor data.
                \item \scriptsize Attributes: Sensor readings (e.g., from a smartphone or wearable device).
                \item \scriptsize Class Labels: “Biking," "Running," etc.
            \end{itemize}
        \end{coloredblock}

        \begin{coloredblock}[blue][\centering\faIcon{envelope}~~~E-Mail Spam Detection][\footnotesize][4.8cm]
            \begin{itemize}
                \item \scriptsize Task: Classify as "spam" or "not spam."
                \item \scriptsize Attributes: Email content, sender, attachements etc.
                \item \scriptsize Class Labels: "Spam" or "Not Spam"
            \end{itemize}
        \end{coloredblock}
        
    \end{minipage}
\end{frame}


%% -------------------------------------------------------------------
%% Closing Slide
\section*{Closing Slide}

\begin{frame}
    % Insert closing slide
    \closingslide{Thank You!}{} % frametitle and subframetitle
\end{frame}


%% -------------------------------------------------------------------
%% Bibliography
\begin{frame}[allowframebreaks]{Bibliography}
  \printbibliography
\end{frame}

\end{document}