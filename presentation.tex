\documentclass[aspectratio=169]{beamer}

%% -------------------------------------------------------------------
%% Theme Configuration
%% -------------------------------------------------------------------
\usetheme[]{iee}

%% -------------------------------------------------------------------
%% Packages
%% -------------------------------------------------------------------
\usepackage{fontawesome5}
\usepackage{pgfplots}
\usepackage{pgf-pie}
\pgfplotsset{compat=1.18} % or your current pgfplots version
\usepackage[table]{xcolor}
\usepackage{csquotes} % Recommended with babel + biblatex
\usepackage{tabularray}
\UseTblrLibrary{booktabs} % Optional, for better lines in tables
\UseTblrLibrary{siunitx}


\usepackage[
  backend      = biber,
  style        = ieee,
  minnames     = 1,
  maxcitenames = 2,
  maxbibnames  = 3,
]{biblatex}
\addbibresource{references.bib}

\DefineBibliographyStrings{german}{
  andothers = {et al.}
}

%% -------------------------------------------------------------------
%% Metadata
%% -------------------------------------------------------------------
% Select language
\usepackage[english]{babel}
% \usepackage[ngerman]{babel}

\title[Short Title]{IEE Template}
\subtitle{for \LaTeX~Presentations}

\author{Robert Gaugl}
\newcommand{\academictitlesauthor}{Ass.Prof. Dipl.-Ing. Dr.techn.}
\newcommand{\coauthors}{}

\date{Presentation Date}

\newcommand{\university}{Graz University of Technology}
\institute{Institute of Electricity Economics and Energy Innovation}
\newcommand{\street}{Inffeldgasse 18}
\newcommand{\postalcode}{8010}
\newcommand{\city}{Graz}
\newcommand{\country}{Austria}

\newcommand{\phone}{+43 316 873 7904}
\newcommand{\mailauthor}{robert.gaugl@tugraz.at}
\newcommand{\urlinstitute}{iee.tugraz.at} % Without https://

\newcommand{\urlinstagram}{instagram.com/iee.tugraz} % Without https://
\newcommand{\urllinkedin}{instagram.com/iee.tugraz} % Without https://

% \logobar{Supported by: \includegraphics[height=1cm]{beamerthemeiee/Logo_IEE_DE.png}} % titlepage sponsors or funding agency

%% -------------------------------------------------------------------
%% Options
%% -------------------------------------------------------------------
% Set globally whether paused content appears transparent or is completely hidden.
% This setting can also be changed locally within individual slides to mix styles.
\setbeamercovered{transparent} % options: transparent or invisible

% Activate navigation symbols in the lower left corner
% \activatenavigationbuttonstrue


% Change how numbers appear when using \SI{}{}
\germanenglish{%
    \sisetup{
        output-decimal-marker = {,},  % use comma as decimal separator
        group-separator = {.},        % use dot as thousand separator
        group-minimum-digits = 4,     % format large numbers
        group-digits=integer,
        detect-mode = true,
        detect-family = false,
    }         
    }{
    \sisetup{
        output-decimal-marker = {.},  % use comma as decimal separator
        group-separator = {~},        % use space as thousand separator
        group-minimum-digits = 4,     % optional: format large numbers
        group-digits=integer,
        detect-mode = true,
        detect-family = false,
    }
}
\DeclareSIUnit \watthour { Wh } %apparent power 

%% -------------------------------------------------------------------
%% Document
%% -------------------------------------------------------------------
\begin{document}

%% -------------------------------------------------------------------
%% Title Slide
\begin{frame}[plain]
    \maketitleslide
\end{frame}


%% -------------------------------------------------------------------
%% Section: Understand the Template for a Consistent Layout
\section{Understand the \textbf{Template} for a Consistent Layout}

\begin{frame}
    \agenda{Agenda}{} % No subtitle and highlights currentvsection
    %\agenda[sections=all]{Agenda}{Overview of lecture structure} % With subtitle and all sections highlighted
\end{frame}


\begin{frame}{Template}
    \framesubtitle{Footnote and Language}

    % Insert a Faded picture (ALWAYS FIRST THING IN A FRAME)
    \insertfadedpicture{15.98cm}{figures/iee_besprechung.png}{Source: Institute of Electricity Economics and Energy Innovation/TU Graz}
    
    \begin{minipage}{0.8\textwidth}  % 80% of the slide width
        \begin{itemize}
            \item To \textcolor{yellow}{\textbf{change}} text/name in \textcolor{yellow}{\textbf{Footnote}} and \textcolor{yellow}{\textbf{Date on Title Slide}}: 
            \vspace{-0.5\topsep}
            \begin{itemize}
                \item The presentation.tex file contains a section at the top labeled Metadata
                \item Modifying the author, date, or institute in this section will automatically update the footnote and date displayed in the presentation
            \end{itemize}
    
            \item Footnote Text
            \vspace{-0.5\topsep}
            \begin{itemize}
                \item The \textcolor{green}{\textbf{default footnote}} text displays the \textcolor{green}{\textbf{institute's name}}
                \item If you want to include your \textcolor{green}{\textbf{lecture title}}, \textcolor{green}{\textbf{presentation topic}}, or other context-specific information in the \textcolor{green}{\textbf{footnote}}, simply enter it in the \textcolor{green}{\textbf{institute field}} within the metadata section
            \end{itemize}
            
            \item Language
            \vspace{-0.5\topsep}
            \begin{itemize}
                \item Ensure that the \textcolor{iee}{\textbf{language}} used throughout the presentation \textcolor{iee}{\textbf{remains consistent}}
                \item For presentations in \textcolor{iee}{\textbf{German}}, set \textcolor{iee}{\textbf{language pack to german}} in the metadata section.
                \item The \textcolor{iee}{\textbf{footnote text}} must \textcolor{iee}{\textbf{match}} the \textcolor{iee}{\textbf{language}} of the presentation
            \end{itemize}
        \end{itemize}
    \end{minipage}
\end{frame}


\begin{frame}{Color palette}
    \framesubtitle{Make use of the the Defined Colors}
    \label{frame:color_palette}
    
    \vspace{1cm}
    \begin{columns}[t]
        \column{.25\textwidth}
            \begin{beamercolorbox}[center,colsep*=4pt]{white}\textbf{Light Shades}\strut\end{beamercolorbox}
            \begin{beamercolorbox}[center,colsep*=4pt]{white}\strut\end{beamercolorbox}
            \begin{beamercolorbox}[center,colsep*=4pt]{greyLight}greenLight\strut\end{beamercolorbox}
            \begin{beamercolorbox}[center,colsep*=4pt]{blueLight}blueLight\strut\end{beamercolorbox}
            \begin{beamercolorbox}[center,colsep*=4pt]{ieeLight}ieeLight\strut\end{beamercolorbox}
            \begin{beamercolorbox}[center,colsep*=4pt]{redLight}redLight\strut\end{beamercolorbox}
            \begin{beamercolorbox}[center,colsep*=4pt]{turquoiseLight}turquoiseLight\strut\end{beamercolorbox}
            \begin{beamercolorbox}[center,colsep*=4pt]{greenLight}greenLight\strut\end{beamercolorbox}
            \begin{beamercolorbox}[center,colsep*=4pt]{yellowLight}yellowLight\strut\end{beamercolorbox}
        \column{.25\textwidth}
            \begin{beamercolorbox}[center,colsep*=4pt]{white}\textbf{Standard}\strut\end{beamercolorbox}
            \begin{beamercolorbox}[center,colsep*=4pt]{white}\strut\end{beamercolorbox}
            \begin{beamercolorbox}[center,colsep*=4pt]{grey}grey\strut\end{beamercolorbox}
            \begin{beamercolorbox}[center,colsep*=4pt]{blue}blue\strut\end{beamercolorbox}
            \begin{beamercolorbox}[center,colsep*=4pt]{iee}iee\strut\end{beamercolorbox}
            \begin{beamercolorbox}[center,colsep*=4pt]{red}red\strut\end{beamercolorbox}
            \begin{beamercolorbox}[center,colsep*=4pt]{turquoise}turquoise\strut\end{beamercolorbox}
            \begin{beamercolorbox}[center,colsep*=4pt]{green}green\strut\end{beamercolorbox}
            \begin{beamercolorbox}[center,colsep*=4pt]{yellow}yellow\strut\end{beamercolorbox}
        \column{.25\textwidth}
            \begin{beamercolorbox}[center,colsep*=4pt]{white}\textbf{Dark Shades}\strut\end{beamercolorbox}
            \begin{beamercolorbox}[center,colsep*=4pt]{white}\strut\end{beamercolorbox}
            \begin{beamercolorbox}[center,colsep*=4pt]{greyDark}greyDark\strut\end{beamercolorbox}
            \begin{beamercolorbox}[center,colsep*=4pt]{blueDark}blueDark\strut\end{beamercolorbox}
            \begin{beamercolorbox}[center,colsep*=4pt]{ieeDark}ieeDark\strut\end{beamercolorbox}
            \begin{beamercolorbox}[center,colsep*=4pt]
            {redDark}redDark\strut\end{beamercolorbox}
            \begin{beamercolorbox}[center,colsep*=4pt]{turquoiseDark}turquoiseDark\strut\end{beamercolorbox}
            \begin{beamercolorbox}[center,colsep*=4pt]{greenDark}greenDark\strut\end{beamercolorbox}
            \begin{beamercolorbox}[center,colsep*=4pt]{yellowDark}yellowDark\strut\end{beamercolorbox}
    \end{columns}
\end{frame}


%% -------------------------------------------------------------------
%% Leverage Boxes to Organize Content
\section{Leverage \textbf{Boxes} to Organize Content}

\begin{frame}
    \agenda{Agenda}{} % No subtitle and highlights currentvsection
    %\agenda[sections=all]{Agenda}{Overview of lecture structure} % With subtitle and all sections highlighted
\end{frame}


\begin{frame}{Boxes}
    \framesubtitle{Organize your Content}

    \begin{coloredblock}[yellow]
        \begin{itemize}
            \item Instead of using lists or enumerations, consider organizing your content into distinct groups for better structure
            \item Boxes are an excellent tool to enhance visual clarity and emphasize group distinctions
        \end{itemize}
    \end{coloredblock}

   % Bottom blocks (side-by-side)
    \begin{columns}
        \begin{column}{0.49\textwidth}
            \begin{coloredblock}[blue][\faIcon{palette}~~~Designing Boxes][\centering][7cm]
                \begin{itemize}
                    \item \textbf{Title Bar}: You can choose between boxes with or without a title bar.
                    \item \textbf{Itemization}: Placing itemized lists inside a box can enhance structure and improve clarity.
                \end{itemize}
            \end{coloredblock}
        \end{column}
        \begin{column}{0.49\textwidth}
            \begin{coloredblock}[blue][\faIcon{lightbulb}~~~Additional Tips][\centering][7cm]
                \begin{itemize}
                    \item Use \textbf{short}, concise \textbf{text} inside the box to maximize clarity.
                    \item Consider \textbf{icons} in the title bar to enhance visual appeal. You can use the icons from the fontawesome package.
                \end{itemize}
            \end{coloredblock}
        \end{column}
    \end{columns}

\end{frame}

\begin{frame}{Boxes}
    \framesubtitle{Standard Boxes with Title}

    \begin{coloredblock}[grey]
        \footnotesize\centering\texttt{\textbackslash begin\{coloredblock\}[Color][Optional:~Title][Optional:~Title~Formatting] [Optional:~Alignment (t,c,b)][Optional:~Height~cm] [Optional:~Width~cm]}
    \end{coloredblock}

    \vspace{-1cm}
    \begin{columns}
        \begin{column}{0.49\textwidth}
    
            \begin{coloredblock}[blue][Blue Block with Title]
                \footnotesize\texttt{\textbackslash begin\{coloredblock\}[blue][Blue Block with Title]}\strut
            \end{coloredblock}
    
            \begin{coloredblock}[yellow][Yellow Block with Title]
                \footnotesize\texttt{\textbackslash begin\{coloredblock\}[yellow][Yellow Block with Title]}\strut
            \end{coloredblock}

            \begin{coloredblock}[red][Red Block with Title]
                \footnotesize\texttt{\textbackslash begin\{coloredblock\}[red][Red Block with Title]}\strut
            \end{coloredblock}
        
        \end{column}
        \begin{column}{0.49\textwidth}
    
            \begin{coloredblock}[iee][IEE Block with Title]
                \footnotesize\texttt{\textbackslash begin\{coloredblock\}[iee][IEE Block with Title]}\strut
            \end{coloredblock}
    
            \begin{coloredblock}[green][Green Block with Title]
                \footnotesize\texttt{\textbackslash begin\{coloredblock\}[green][Green Block with Title]}\strut
            \end{coloredblock}
    
            \begin{coloredblock}[grey][Grey Block with Title]
                \footnotesize\texttt{\textbackslash begin\{coloredblock\}[grey][Grey Block with Title]}\strut
            \end{coloredblock}
        \end{column}
    \end{columns}

\end{frame}

\begin{frame}{Boxes}
    \framesubtitle{Standard Boxes without Title}

    \begin{coloredblock}[grey]
        \footnotesize\centering\texttt{\textbackslash begin\{coloredblock\}[Color][Optional:~Title][Optional:~Title~Formatting] [Optional:~Alignment (t,c,b)][Optional:~Height~cm] [Optional:~Width~cm]}
    \end{coloredblock}

    \vspace{-1cm}
    \begin{columns}
        \begin{column}{0.49\textwidth}

            \begin{coloredblock}[blue]
                \footnotesize\footnotesize\texttt{\textbackslash begin\{coloredblock\}[blue]}\strut
            \end{coloredblock}
    
            \begin{coloredblock}[yellow]
                \footnotesize\texttt{\textbackslash begin\{coloredblock\}[yellow]}\strut
            \end{coloredblock}
    
            \begin{coloredblock}[red]
                \footnotesize\texttt{\textbackslash begin\{coloredblock\}[red]}\strut
            \end{coloredblock}

        \end{column}
        \begin{column}{0.49\textwidth}
        
            \begin{coloredblock}[iee]
                \footnotesize\texttt{\textbackslash begin\{coloredblock\}[iee]}\strut
            \end{coloredblock}
    
            \begin{coloredblock}[green]
                \footnotesize\texttt{\textbackslash begin\{coloredblock\}[green]}\strut
            \end{coloredblock}
    
            \begin{coloredblock}[grey]
                \footnotesize\texttt{\textbackslash begin\{coloredblock\}[grey]}\strut
            \end{coloredblock}
        
        \end{column}
    \end{columns}

    \centering
    \begin{minipage}[t]{0.49\textwidth}
        \begin{coloredblock}[turquoise]
                \footnotesize\texttt{\textbackslash begin\{coloredblock\}[turquoise]}\strut
        \end{coloredblock}
    \end{minipage}
\end{frame}


\begin{frame}{Boxes}
    \framesubtitle{Dark Boxes with Title}

    \begin{coloredblock}[grey]
        \footnotesize\centering\texttt{\textbackslash begin\{coloredblockdark\}[Color][Optional:~Title][Optional:~Title~Formatting] [Optional:~Alignment (t,c,b)][Optional:~Height~cm] [Optional:~Width~cm]}
    \end{coloredblock}

    \vspace{-1cm}
    \begin{columns}
        \begin{column}{0.49\textwidth}
    
            \begin{coloredblockdark}[blue][Dark Blue Block with Title]
                \footnotesize\texttt{\textbackslash begin\{coloredblockdark\}[blue][Dark Blue Block with Title]}\strut
            \end{coloredblockdark}
    
            \begin{coloredblockdark}[yellow][Dark Yellow Block with Title]
                \footnotesize\texttt{\textbackslash begin\{coloredblockdark\}[yellow]\allowbreak [Dark Yellow Block with Title]}\strut
            \end{coloredblockdark}
    
            \begin{coloredblockdark}[red][Dark Red Block with Title]
                \footnotesize\texttt{\textbackslash begin\{coloredblockdark\}[red][Dark Red Block with Title]}\strut
            \end{coloredblockdark}
        
        \end{column}
        \begin{column}{0.49\textwidth}
    
            \begin{coloredblockdark}[iee][Dark IEE Block with Title]
                \footnotesize\texttt{\textbackslash begin\{coloredblockdark\}[iee][Dark IEE Block with Title]}\strut
            \end{coloredblockdark}
    
            \begin{coloredblockdark}[green][Dark Green Block with Title]
                \footnotesize\texttt{\textbackslash begin\{coloredblockdark\}[green]\allowbreak [Dark Green Block with Title]}\strut
            \end{coloredblockdark}
    
            \begin{coloredblockdark}[grey][Dark Grey Block with Title]
                \footnotesize\texttt{\textbackslash begin\{coloredblockdark\}[grey][Dark Grey Block with Title]}\strut
            \end{coloredblockdark}
        
        \end{column}
    \end{columns}
\end{frame}

\begin{frame}{Boxes}
    \framesubtitle{Dark Boxes without Title}

    \begin{coloredblock}[grey]
        \footnotesize\centering\texttt{\textbackslash begin\{coloredblockdark\}[Color][Optional:~Title][Optional:~Title~Formatting] [Optional:~Alignment (t,c,b)][Optional:~Height~cm] [Optional:~Width~cm]}
    \end{coloredblock}

    \vspace{-1cm}
    \begin{columns}
        \begin{column}{0.49\textwidth}

            \begin{coloredblockdark}[blue]
                \footnotesize\texttt{\textbackslash begin\{coloredblockdark\}[blue]}\strut
            \end{coloredblockdark}
    
            \begin{coloredblockdark}[yellow]
                \footnotesize\texttt{\textbackslash begin\{coloredblockdark\}[yellow]}\strut
            \end{coloredblockdark}
    
            \begin{coloredblockdark}[red]
                \footnotesize\texttt{\textbackslash begin\{coloredblockdark\}[red]}\strut
            \end{coloredblockdark}

        \end{column}
        \begin{column}{0.49\textwidth}
        
            \begin{coloredblockdark}[iee]
                \footnotesize\texttt{\textbackslash begin\{coloredblockdark\}[iee]}\strut
            \end{coloredblockdark}
    
            \begin{coloredblockdark}[green]
                \footnotesize\texttt{\textbackslash begin\{coloredblockdark\}[green]}\strut
            \end{coloredblockdark}
    
            \begin{coloredblockdark}[grey]
                \footnotesize\texttt{\textbackslash begin\{coloredblockdark\}[grey]}\strut
            \end{coloredblockdark}
        
        \end{column}
    \end{columns}

    \centering
    \begin{minipage}[t]{0.49\textwidth}
        \begin{coloredblockdark}[turquoise]
                \footnotesize\texttt{\textbackslash begin\{coloredblockdark\}[turquoise]}\strut
        \end{coloredblockdark}
        
    \end{minipage}
\end{frame}

\begin{frame}{Boxes}
    \framesubtitle{Boxes with Icon and with Title}

    \begin{coloredblock}[grey]
        \footnotesize\centering\texttt{\textbackslash begin\{coloredblock\}[Color][Optional:~Title][Optional:~Title~Size] [Optional:~Icon][Optional:~Side~of~icon (l,r)][Optional:~Alignment (t,c,b)][Optional:~Height~cm][Optional:~Width~Iconblock~cm]}
    \end{coloredblock}

    \vspace{-1cm}
    \begin{columns}
        \begin{column}{0.49\textwidth}

            \begin{coloredblockicon}[blue][Blue Box with Title and Icon][][\large\faIcon{bolt}]
                \footnotesize\texttt{\textbackslash begin\{coloredblock\}[blue][Blue Box with Title and Icon][][\textbackslash faIcon\{bolt\}]}\strut
            \end{coloredblockicon}
    
            \begin{coloredblockicon}[yellow][Yellow Box with Title and Icon][][\large\faIcon{bolt}]
                \footnotesize\texttt{\textbackslash begin\{coloredblock\}[yellow] [Yellow Box with Title and Icon][][\textbackslash faIcon\{bolt\}]}\strut
            \end{coloredblockicon}

        \end{column}
        \begin{column}{0.49\textwidth}
        
            \begin{coloredblockicon}[iee][IEE Box with Title and Icon][][\large\faIcon{bolt}][r]
                \footnotesize\texttt{\textbackslash begin\{coloredblock\}[iee][IEE Box with Title and Icon][][\textbackslash faIcon\{bolt\}][r]}\strut
            \end{coloredblockicon}
    
            \begin{coloredblockicon}[green][Green Box with Title and Icon][][\large\faIcon{bolt}][r]
                \footnotesize\texttt{\textbackslash begin\{coloredblock\}[green] [Green Box with Title and Icon][][\textbackslash faIcon\{bolt\}][r]}\strut
            \end{coloredblockicon}
        
        \end{column}
    \end{columns}

\end{frame}

\begin{frame}{Boxes}
    \framesubtitle{Boxes with Icon and without Title}

    \begin{coloredblock}[grey]
        \footnotesize\centering\texttt{\textbackslash begin\{coloredblock\}[Color][Optional:~Title][Optional:~Title~Size] [Optional:~Icon][Optional:~Side~of~icon (l,r)][Optional:~Alignment (t,c,b)][Optional:~Height~cm][Optional:~Width~Iconblock~cm]}
    \end{coloredblock}
        

    \vspace{-1cm}
    \begin{columns}
        \begin{column}{0.49\textwidth}

            \begin{coloredblockicon}[blue][][][\large\faIcon{bolt}][l][2cm][2cm]
                \footnotesize\texttt{\textbackslash begin\{coloredblockicon\}[blue]}\strut
            \end{coloredblockicon}
    
            \begin{coloredblockicon}[yellow][][][\large\faIcon{bolt}][l][2cm][2cm]
                \footnotesize\texttt{\textbackslash begin\{coloredblockicon\}[yellow]}\strut
            \end{coloredblockicon}
    
            \begin{coloredblockicon}[red][][][\large\faIcon{bolt}][l][2cm][2cm]
                \footnotesize\texttt{\textbackslash begin\{coloredblockicon\}[red]}\strut
            \end{coloredblockicon}

        \end{column}
        \begin{column}{0.49\textwidth}
        
            \begin{coloredblockicon}[iee][][][\large\faIcon{bolt}][r][2cm][2cm]
                \footnotesize\texttt{\textbackslash begin\{coloredblockicon\}[iee]}\strut
            \end{coloredblockicon}
    
            \begin{coloredblockicon}[green][][][\large\faIcon{bolt}][r][2cm][2cm]
                \footnotesize\texttt{\textbackslash begin\{coloredblockicon\}[green]}\strut
            \end{coloredblockicon}
    
            \begin{coloredblockicon}[grey][][][\large\faIcon{bolt}][r][2cm][2cm]
                \footnotesize\texttt{\textbackslash begin\{coloredblockicon\}[grey]}\strut
            \end{coloredblockicon}
        
        \end{column}
    \end{columns}

    \centering
    \begin{minipage}[t]{0.49\textwidth}
        \begin{coloredblockicon}[turquoise][][][\large\faIcon{bolt}][l][2cm][2cm]
            \vspace{0.2cm}
            \footnotesize\texttt{\textbackslash begin\{coloredblockicon\}[turquoise]}\strut
        \end{coloredblockicon}
        
    \end{minipage}
\end{frame}


%% -------------------------------------------------------------------
%% Use the Typeface to Enhance Readability
\section{Use the \textbf{Typeface} to Enhance Readability}

\begin{frame}
    \agenda{Agenda}{} % No subtitle and highlights currentvsection
    %\agenda[sections=all]{Agenda}{Overview of lecture structure} % With subtitle and all sections highlighted
\end{frame}

\begin{frame}{Typeface}
    \framesubtitle{Font and Styling}

    \begin{coloredblock}[blue][\faIcon{highlighter}~~~Highlighting][\centering]
        \begin{itemize}
            \item Use \textbf{bold text} to \textbf{emphasize} key points
            \item For a subtler emphasis, you can use \textit{italics}
            \item Avoid using \underline{underlines} for emphasis, as underlined text is commonly interpreted as a hyperlink and may lead to confusion
        \end{itemize}
    \end{coloredblock}

    \vspace{0.2cm}

    \begin{coloredblock}[yellow][\faIcon{fill-drip}~~~Color][\centering]
            \begin{itemize}
                \item Make the most of the \textbf{default colors} available (IEE, blue, yellow, green, red and grey)
                \item To \textbf{color text}, use \texttt{\textbackslash textcolor\{color\{Text\}\}}
                \item To draw attention, use \textbf{yellow} for \textbf{moderate emphasis} and \textbf{red} for \textbf{strong emphasis}
            \end{itemize}
    \end{coloredblock}
    
\end{frame}

\begin{frame}{Typeface}
    \framesubtitle{Available Font Sizes}
    \vfill
    \begin{columns}
        \column{.3\textwidth}%
        \centering
        \thefontsize[TINY]\TINY
        \thefontsize[Tiny]\Tiny
        \thefontsize[myFootnotesize]\myFootnotesize
        \thefontsize[tiny]\tiny
        \thefontsize[scriptsize]\scriptsize
        \thefontsize[footnotesize]\footnotesize
        \thefontsize[small]\small
        \thefontsize[normalsize]\normalsize
        \thefontsize[large]\large
    
        \column{.65\textwidth}%
        \centering
        \thefontsize[Large]\Large
        \thefontsize[LARGE]\LARGE
        \thefontsize[huge]\huge
        \thefontsize[Huge]\Huge
    \end{columns}
    \vfill
\end{frame}


%% -------------------------------------------------------------------
%% Reveal your content step-by-step withn Overlays
\section{Reveal your content step-by-step with \textbf{Overlays}}

\begin{frame}
    \agenda{Agenda}{} % No subtitle and highlights currentvsection
    %\agenda[sections=all]{Agenda}{Overview of lecture structure} % With subtitle and all sections highlighted
\end{frame}

\begin{frame}{Overlay}
    \framesubtitle{Reveal your content step-by-step}

    \begin{coloredblockicon}[yellow][][][\large\faIcon{exclamation}]
        \begin{itemize}
            \item \footnotesize Enhance your presentation by \textbf{revealing} content \textbf{step by step} to \textbf{guide audience attention}.
            \item \footnotesize \textbf{No need to duplicate slides} — \textbf{overlays} handle it, and the slide number stays the same.
        \end{itemize}
    \end{coloredblockicon}

    \pause
    \vspace{-0.5cm}
    \begin{columns}
        \begin{column}{0.49\textwidth}
            \begin{coloredblock}[turquoise][\faIcon{pause}~~~Pause Command][\small\centering][10cm]
                \begin{itemize}
                    \item \footnotesize Insert \textbf{\texttt{\textbackslash pause}} at the point where you want the next item to appear.
                    \item \footnotesize Everything \textbf{before} \texttt{\textbackslash pause} is shown on the current slide.
                    \item \footnotesize Everything \textbf{after} \texttt{\textbackslash pause} is shown only on the next reveal step.
                    \item \footnotesize You can use \textbf{multiple \texttt{\textbackslash pause}} commands in one frame to reveal items one at a time.
                \end{itemize}
            \end{coloredblock}
        \end{column}
        \begin{column}{0.49\textwidth}
            \setbeamercovered{invisible}
            \pause
                \begin{coloredblock}[green][\faIcon{code-branch}~~~Transparent or Invisible][\small\centering][10cm]
                    \begin{itemize}
                       \item \footnotesize Unrevealed content can either be \textbf{transparent} or \textbf{invisible} until it is revealed.
                        \item \footnotesize You control this using 
                        \vspace{-0.4cm}
                        \begin{itemize}
                            \item \texttt{\textbackslash setbeamercovered\{transparent\}} or
                            \item \texttt{\textbackslash setbeamercovered\{invisible\}}.
                        \end{itemize}
                        \item \footnotesize \textbf{Change} the \textbf{setting} in the \textbf{preamble} to \textbf{apply globally}, or \textbf{use them inside a frame} to \textbf{apply} only to \textbf{specific content}.
                    \end{itemize}
                \end{coloredblock}
            \end{column}
        \end{columns}

\end{frame}


%% -------------------------------------------------------------------
%% Visualize your Numbers with Plots
\section{Visualize your Numbers with \textbf{Plots}}

\begin{frame}
    \agenda{Agenda}{} % No subtitle and highlights currentvsection
    %\agenda[sections=all]{Agenda}{Overview of lecture structure} % With subtitle and all sections highlighted
\end{frame}

\begin{frame}{Plots}
    \framesubtitle{Visualize your Numbers}

    \vspace{-1cm}
    \begin{columns}
        \begin{column}{0.49\textwidth}
            \begin{coloredblockicon}[blue][][][\large\faIcon{chart-line}][l][2.3cm][2.3cm]
                \footnotesize
                Set \textbf{axis line style} to \textbf{black} to add a black \textbf{border} around the plot area
            \end{coloredblockicon}
            \vspace{-0.1cm}
            \begin{coloredblockicon}[yellow][][][\large\faIcon{list-ol}][l][2.3cm][2.3cm]
                \footnotesize
                Include \textbf{axis labels} with descriptions and \textbf{units} and include thousand separator
            \end{coloredblockicon}
            \vspace{-0.1cm}
            \begin{coloredblockicon}[grey][][][\large\faIcon[regular]{square}][l][2.3cm][2.3cm]
                \footnotesize
                Add a \textbf{black border} around the \textbf{sections} of the bar chart
            \end{coloredblockicon}
            \vspace{-0.1cm}
            \begin{coloredblockicon}[green][][][\large\faIcon{chart-bar}][l][2.3cm][2.3cm]
                \footnotesize
                Include a \textbf{legend} to explain the chart's data categories
            \end{coloredblockicon}
            \vspace{-0.1cm}
            \begin{coloredblock}[iee][\centering \faIcon{star}~~~Optional Improvements][\footnotesize][3.9cm]
                    \begin{itemize}
                        \item \footnotesize Add grey grid lines to the background
                        \item \footnotesize Include caption in italics at the bottom, complete with figure numbering
                    \end{itemize}
            \end{coloredblock}
        \end{column}
        \begin{column}{0.49\textwidth}
    % Left side of the page
            \vspace{1cm}
            \tiny
            \begin{overprint}
                \onslide<1>\begin{figure}[htbp]
                    % Figure generated seperately for faster compilation time.
                    \includegraphics[width=\linewidth]{figures/installed_capacity_2022-2025.pdf}
                    \caption{\centering Installed capacity per power plant type in Wakanda from 2022-2025.}
                    \label{fig:installed_capacity}
            \end{figure}
    
            \onslide<2>\begin{figure}[htbp]
                % Figure generated seperately for faster compilation time. 
                 \includegraphics[width=\linewidth]{figures/demand_rp.pdf}  % path to your PDF file
                \caption{\centering Demand per representative periode in Wakanda.}
                \label{fig:demand}
            \end{figure}
     
            \onslide<3>\vspace{-1cm}\begin{figure}[htbp]
                % Figure generated seperately for faster compilation time. 
                \includegraphics[width=\linewidth]
                {figures/installed_capacity_2023_percent.pdf}  % path to your PDF file
                \caption{\centering Shares of installed capacity per power plant type in Wakanda 2023.}
                \label{fig:installed_capacity_2023}
            \end{figure}
            \end{overprint}
        \end{column}
    \end{columns}

    \addsource{Source: Energy Report, Statistics Wakanda, 2023}
\end{frame}


\begin{frame}{Plots}
    \framesubtitle{Colors for Power Plants}

    \begin{table}[htbp]
        \scalebox{0.94}{
            % ==============================================================================
% Power Plant Colors Table
% ==============================================================================
% This file defines a color-coded table of power plant types using tabularray.
% It can be compiled standalone (for preview/export) or input into a larger LaTeX
% document (e.g., a Beamer presentation).
%
% To compile standalone, set \standalonetrue below.
% To use via \input{}, ensure \standalonefalse is set (default).
% ==============================================================================

\newif\ifstandalone
% \standalonetrue    % Uncomment this line to compile standalone
\standalonefalse     % Default for input into other documents

\ifstandalone
    \documentclass{standalone}
    \usepackage[table]{xcolor}
    \usepackage{tabularray}
    \UseTblrLibrary{booktabs}
    \UseTblrLibrary{siunitx}
    \usepackage{graphicx}
    \usepackage{tikz}
    \usepackage{sfmath}
    \renewcommand{\familydefault}{\sfdefault}

    % Load color definitions (always)
    % TU Graz Colors
\definecolor{tug}{HTML}{F70146}

% IEE Color
\definecolor{iee}{HTML}{008080}
\colorlet{main}{iee}

% PowerPoint Palette
\definecolor{fore}{HTML}{0F0F0F}
\definecolor{back}{HTML}{FFFFFF}
\definecolor{dark}{HTML}{3B5A70}
\definecolor{lite}{HTML}{A6A6A6}
\definecolor{head}{HTML}{245B78}
\definecolor{body}{HTML}{E2E9ED}
\definecolor{urlA}{HTML}{0066D8}
\definecolor{urlB}{HTML}{6C2F91}
\colorlet{colA}{iee}
\colorlet{colB}{tug}
\definecolor{colC}{HTML}{6BA3A3}
\definecolor{colD}{HTML}{2E4172}
\definecolor{colE}{HTML}{78BE73}
\definecolor{colF}{HTML}{D58E00}
\colorlet{grey}{lite}
\colorlet{default}{dark}

% Alternate color names
\colorlet{blue}{dark}
\colorlet{turquoise}{colC}
\colorlet{green}{colE}
\colorlet{yellow}{colF}
\colorlet{red}{colB}

% Light Shades
\definecolor{greyLight}{HTML}{EDEDED}
\definecolor{blueLight}{HTML}{D3DFE8}
\definecolor{ieeLight}{HTML}{E1EDED}
\definecolor{redLight}{HTML}{FFCBDA}
\definecolor{turquoiseLight}{HTML}{E1EDED}
\definecolor{greenLight}{HTML}{E4F2E3}
\definecolor{yellowLight}{HTML}{FFEBC4}

% Dark Shades
\definecolor{greyDark}{HTML}{0F0F0F}
\definecolor{blueDark}{HTML}{1D2D38}
\definecolor{ieeDark}{HTML}{004040}
\definecolor{redDark}{HTML}{7C0023}
\definecolor{turquoiseDark}{HTML}{345353}
\definecolor{greenDark}{HTML}{346830}
\definecolor{yellowDark}{HTML}{6A4700}

% Power Plant Colors
\definecolor{oilColor}{RGB}{16, 47, 64}
\definecolor{coalColor}{RGB}{127, 127, 127}
\definecolor{gasColor}{RGB}{191, 191, 191}
\definecolor{otherNonResColor}{RGB}{221, 110, 56}
\definecolor{nuclearColor}{RGB}{236, 96, 95}
\definecolor{biomassColor}{RGB}{198, 70, 59}
\definecolor{rorColor}{RGB}{0, 176, 240}
\definecolor{pvColor}{RGB}{255, 192, 0}
\definecolor{windColor}{RGB}{146, 208, 80}
\definecolor{windOffColor}{RGB}{30, 143, 79}
\definecolor{otherResColor}{RGB}{185, 207, 222}
\definecolor{storageHydroColor}{RGB}{75, 172, 198}
\definecolor{pumpedStorageColor}{RGB}{59, 90, 112}
\definecolor{bessColor}{RGB}{112, 48, 160}
\definecolor{otherStorageColor}{RGB}{76, 133, 123}

    \begin{document}
\fi

\footnotesize
\centering
\begin{tblr}{
    colspec={lllrrrrc},
    hline{1-2}={solid,1.5pt},
    hline{3-Y}={grey},
    hline{Z}={solid,1.5pt},
    row{odd}={grey!20},
    row{1}={white,font=\bfseries},
    cell{2}{8}={bg=oilColor},
    cell{3}{8}={bg=coalColor},
    cell{4}{8}={bg=gasColor},
    cell{5}{8}={bg=otherNonResColor},
    cell{6}{8}={bg=nuclearColor},
    cell{7}{8}={bg=biomassColor},
    cell{8}{8}={bg=rorColor},
    cell{9}{8}={bg=pvColor},
    cell{10}{8}={bg=windColor},
    cell{11}{8}={bg=windOffColor},
    cell{12}{8}={bg=otherResColor},
    cell{13}{8}={bg=storageHydroColor},
    cell{14}{8}={bg=pumpedStorageColor},
    cell{15}{8}={bg=bessColor},
    cell{16}{8}={bg=otherStorageColor},
}
    \textbf{Power Plant Type} & \textbf{German Name} & \textbf{LaTeX~Name} & \textbf{R} & \textbf{G} & \textbf{B} & \textbf{Hex} & \textbf{Color} \\
    Oil & Öl & \texttt{\textbackslash oilColor} & 16 & 47 & 64 & 102F40 & \\
    Coal & Kohle & \texttt{\textbackslash coalColor} & 127 & 127 & 127 & 7F7F7F &  \\
    Gas & Gas & \texttt{\textbackslash gasColor} & 191 & 191 & 191 & BFBFBF &  \\
    Other non-RES & Sonstige nicht-EE & \texttt{\textbackslash otherNonResColor} & 221 & 110 & 56 & DD6E38 &  \\
    Nuclear & Nuklear & \texttt{\textbackslash nuclearColor} & 236 & 96 & 95 & EC605F &  \\
    Biomass & Biomasse & \texttt{\textbackslash biomassColor} & 198 & 70 & 59 & C6463B &  \\
    Run-of-River & Laufwasserkraft & \texttt{\textbackslash rorColor} & 0 & 176 & 240 & 00B0F0 &  \\
    Solar/PV & Solar/PV & \texttt{\textbackslash pvColor} & 255 & 192 & 0 & FFC000 &  \\
    Wind-onshore & Wind-onshore & \texttt{\textbackslash windColor} & 146 & 208 & 80 & 92D050 &  \\
    Wind-offshore & Wind-offshore & \texttt{\textbackslash windOffColor} & 30 & 143 & 79 & 1E8F4F &  \\
    Other RES & Sonstige EE & \texttt{\textbackslash otherResColor} & 185 & 207 & 222 & B9CFDE &  \\
    Storage Hydro & Wasserspeicher & \texttt{\textbackslash storageHydroColor} & 75 & 172 & 198 & 4BACC6 &  \\
    Pumped Storage & Pumpspeicher & \texttt{\textbackslash pumpedStorageColor} & 59 & 90 & 112 & 3B5A70 &  \\
    BESS & Batteriespeicher & \texttt{\textbackslash bessColor} & 112 & 48 & 160 & 7030A0 &  \\
    Other Storages & Sonstige Speicher & \texttt{\textbackslash otherStorageColor} & 76 & 133 & 123 & 4C857B &  \\
\end{tblr}

\ifstandalone
    \end{document}
\fi

        }
    \end{table}

\end{frame}


%% -------------------------------------------------------------------
%% Structure your Data with Clear Tables
\section{\textbf{Structure} your \textbf{Data} with Clear \textbf{Tables}}

\begin{frame}
    \agenda{Agenda}{} % No subtitle and highlights currentvsection
    %\agenda[sections=all]{Agenda}{Overview of lecture structure} % With subtitle and all sections highlighted
\end{frame}

\begin{frame}{Tables}
    \framesubtitle{Clear and Well-Designed Tables}

    \vspace{-.7cm}
    \begin{columns}
        \begin{column}{0.49\textwidth}
            \begin{coloredblock}[turquoise][\faIcon{table}~~~Table Design][\footnotesize\centering]
                \vspace{0.2cm}
                \begin{tugitemize}
                    \item \footnotesize \textbf{Bold headers} for clarity and emphasis
                    \item \textbf{Black vertical lines} for \textbf{header} and \textbf{bottom} only (1.5~pt) 
                    \item \textbf{Grey vertical lines} (1~pt) for the \textbf{rest}
                    \item \footnotesize \textbf{Avoid horizontal lines} (unless necessary)
                    \item \footnotesize \textbf{Use gray!20} as \textbf{background} for \textbf{odd rows}
                \end{tugitemize}
            \end{coloredblock}
            \vspace{0.2cm}
            \begin{coloredblock}[yellow][\faIcon{glasses}~~~Data Readability][\footnotesize\centering]
                \vspace{0.2cm}
                \begin{tugitemize}
                    \item \footnotesize \textbf{Align numbers by decimal separator} for better readability
                    \item \footnotesize \textbf{Show only meaningful decimals} – no clutter!
                    \item \footnotesize \textbf{Use consistent units} and place them in the header when possible
                    \item \footnotesize \textbf{Highlight} the numbers you want to emphasize
                \end{tugitemize}
            \end{coloredblock}
        \end{column}
        \begin{column}{0.49\textwidth}
            \vspace{2cm}            
            \begin{table}[htbp]
                % ==============================================================================
% Power Plant Colors Table
% ==============================================================================
% This file defines a color-coded table of power plant types using tabularray.
% It can be compiled standalone (for preview/export) or input into a larger LaTeX
% document (e.g., a Beamer presentation).
%
% To compile standalone, set \standalonetrue below.
% To use via \input{}, ensure \standalonefalse is set (default).
% ==============================================================================

\newif\ifstandalone
% \standalonetrue    % Uncomment this line to compile standalone
\standalonefalse     % Default for input into other documents

\ifstandalone
    \documentclass{standalone}
    \usepackage[table]{xcolor}
    \usepackage{tabularray}
    \UseTblrLibrary{booktabs}
    \UseTblrLibrary{siunitx}
    \usepackage{graphicx}
    \usepackage{tikz}
    \usepackage{sfmath}
    \renewcommand{\familydefault}{\sfdefault}

    % Load color definitions (always)
    % TU Graz Colors
\definecolor{tug}{HTML}{F70146}

% IEE Color
\definecolor{iee}{HTML}{008080}
\colorlet{main}{iee}

% PowerPoint Palette
\definecolor{fore}{HTML}{0F0F0F}
\definecolor{back}{HTML}{FFFFFF}
\definecolor{dark}{HTML}{3B5A70}
\definecolor{lite}{HTML}{A6A6A6}
\definecolor{head}{HTML}{245B78}
\definecolor{body}{HTML}{E2E9ED}
\definecolor{urlA}{HTML}{0066D8}
\definecolor{urlB}{HTML}{6C2F91}
\colorlet{colA}{iee}
\colorlet{colB}{tug}
\definecolor{colC}{HTML}{6BA3A3}
\definecolor{colD}{HTML}{2E4172}
\definecolor{colE}{HTML}{78BE73}
\definecolor{colF}{HTML}{D58E00}
\colorlet{grey}{lite}
\colorlet{default}{dark}

% Alternate color names
\colorlet{blue}{dark}
\colorlet{turquoise}{colC}
\colorlet{green}{colE}
\colorlet{yellow}{colF}
\colorlet{red}{colB}

% Light Shades
\definecolor{greyLight}{HTML}{EDEDED}
\definecolor{blueLight}{HTML}{D3DFE8}
\definecolor{ieeLight}{HTML}{E1EDED}
\definecolor{redLight}{HTML}{FFCBDA}
\definecolor{turquoiseLight}{HTML}{E1EDED}
\definecolor{greenLight}{HTML}{E4F2E3}
\definecolor{yellowLight}{HTML}{FFEBC4}

% Dark Shades
\definecolor{greyDark}{HTML}{0F0F0F}
\definecolor{blueDark}{HTML}{1D2D38}
\definecolor{ieeDark}{HTML}{004040}
\definecolor{redDark}{HTML}{7C0023}
\definecolor{turquoiseDark}{HTML}{345353}
\definecolor{greenDark}{HTML}{346830}
\definecolor{yellowDark}{HTML}{6A4700}

% Power Plant Colors
\definecolor{oilColor}{RGB}{16, 47, 64}
\definecolor{coalColor}{RGB}{127, 127, 127}
\definecolor{gasColor}{RGB}{191, 191, 191}
\definecolor{otherNonResColor}{RGB}{221, 110, 56}
\definecolor{nuclearColor}{RGB}{236, 96, 95}
\definecolor{biomassColor}{RGB}{198, 70, 59}
\definecolor{rorColor}{RGB}{0, 176, 240}
\definecolor{pvColor}{RGB}{255, 192, 0}
\definecolor{windColor}{RGB}{146, 208, 80}
\definecolor{windOffColor}{RGB}{30, 143, 79}
\definecolor{otherResColor}{RGB}{185, 207, 222}
\definecolor{storageHydroColor}{RGB}{75, 172, 198}
\definecolor{pumpedStorageColor}{RGB}{59, 90, 112}
\definecolor{bessColor}{RGB}{112, 48, 160}
\definecolor{otherStorageColor}{RGB}{76, 133, 123}

    \begin{document}
\fi
\centering
\small
\begin{tblr}{
      colspec = {
            l 
            S[table-format=4.1] % S for decimal-aligned numeric columns
            S[table-format=3.1] % S for decimal-aligned numeric columns
            S[table-format=3.1] % S for decimal-aligned numeric columns
            S[table-format=3.1] % S for decimal-aligned numeric columns
      },
      hline{1-2}={solid,1.5pt}, % Header with black lines on top and bottom
      hline{3-Y}={grey}, % Grey lines between rows
      hline{Z}={solid,1.5pt}, % Black line at the end
      row{odd}={grey!20}, % Background color odd rows
      % row{even}={white}, % Background color even rows
      row{1}={white,font=\bfseries}, % Bold font for header
      cell{4}{2}={yellowLight, cmd=\textbf}, % Highlight cell (has to be after row{odd}={grey!20})
      cell{4}{3}={yellowLight, cmd=\textbf}, % Highlight cell (has to be after row{odd}={grey!20})
    }
    \textbf{Power Plant Type} & \textbf{2022} & \textbf{2023} & \textbf{2024} & \textbf{2025} \\
    Coal           & 25.3 & 20.6 & 15.9 & 10.5 \\
    Gas            & 75.3 & 70.5 & 60.2 & 55.6 \\
    Run-of-River   & 40.3 & 45.0 & 50.0 & 50.0 \\
    PV             & 50.5 & 70.5 & 90.7 & 120.0 \\
    Wind           & 80.0 & 90.2 & 110.3 & 130.4 \\
    Storage Hydro  & 10.7 & 20.0 & 30.7 & 40.2 \\
\end{tblr}

\ifstandalone
    \end{document}
\fi

                \caption{\centering Installed capacity per power plant type in Wakanda from 2022--2025 in MW.}
            \end{table}
        \end{column}
    \end{columns}

    \addsource{Source: Energy Report, Statistics Wakanda, 2023}
    
\end{frame}


%% -------------------------------------------------------------------
%% Additional Tips to Improve your Design
\section{Additional \textbf{Tips} to Improve your Design}

\begin{frame}
    \agenda{Agenda}{} % No subtitle and highlights currentvsection
    %\agenda[sections=all]{Agenda}{Overview of lecture structure} % With subtitle and all sections highlighted
\end{frame}

\begin{frame}{Additional Tips}
    \framesubtitle{Improve your Design}

    \begin{coloredblock}[iee][\faIcon{dog}~~~Icon][\centering]
        \begin{itemize}
            \item Incorporate \textbf{icons} to enhance \textbf{visual clarity and aid }quick \textbf{comprehension}
            \item The \textbf{recommended option} in \LaTeX~is to use icons from the \textbf{fontawesome package}. You can find all \textbf{available icons} in the \href{https://mirror.easyname.at/ctan/fonts/fontawesome5/doc/fontawesome5.pdf}{\textbf{docs}}. Add an attribution on the \hyperlink{frame:closing_slide}{closing slide}.

            \item When using \textbf{third-party icons}:
            \vspace{-0.5\topsep}
            \begin{itemize}
                \item Ensure the icons follow a \textbf{consistent design} style
                \item Verify that you have the appropriate \textbf{usage rights} for the icons
            \end{itemize}
        \end{itemize}
    \end{coloredblock}

        \begin{coloredblock}[yellow][\faIcon{image}~~~Pictures][\centering]
            \begin{itemize}
                \item Always \textbf{credit} the \textbf{source} of your pictures. Use the function \texttt{\textbackslash addsource\{\}}.
                \item Ensure all images are \textbf{crisp} and \textbf{sharp} for \textbf{optimal quality}
            \end{itemize}
        \end{coloredblock}
\end{frame}

\begin{frame}{Faded Image}
    \framesubtitle{Add a Nice Visual Touch}

    \vspace{.6cm}
    \insertfadedpicture{15.98cm}{figures/iee_besprechung.png}{Source: Institute of Electricity Economics and Energy Innovation/TU Graz}

        \begin{coloredblock}[blue][\faIcon{image}~~~Faded Image][\centering ][][][.75\textwidth]
            \begin{itemize}
               \item To \textbf{add} a \textbf{faded image} as a background to your slide, use the \textbf{following command}:
                \item[] \begin{center}\footnotesize\texttt{\textbackslash insertfadedpicture\{inset from right\}\{image path\}\{Optional: caption text\}}\end{center}
                \begin{itemize}
                    \item \textbf{Inset from right}: Defines how far the image should be shifted in from the right edge. (Ensure the image is large enough.)
                    \item \textbf{Image path}: Specifies the file path to the image.
                    \item \textbf{Caption text}: Provides the text (e.g., image source) to be displayed in the bottom-right corner.
                \end{itemize}
                \item This \textbf{must be included at the beginning} of your \textbf{frame}.
            \end{itemize}     
        \end{coloredblock}

\end{frame}


\begin{frame}{Lists}
    \framesubtitle{Different Styles of Lists}

    \begin{coloredblock}[yellow]
        \centering
        There are \textbf{different styles} of \textbf{lists} that you can choose from.
    \end{coloredblock}

    \vspace{-0.5cm}

    % Bottom blocks (side-by-side)
    \begin{columns}
        \begin{column}{0.49\textwidth}
            \begin{coloredblock}[blue][\texttt{\textbf{\textbackslash begin\{itemize\}}}:][\footnotesize\centering][][4.5cm]
                \begin{itemize}
                    \item Item level 1
                    \begin{itemize}
                        \item Item level 2
                        \begin{itemize}
                            \item Item level 3
                        \end{itemize}
                    \end{itemize}
                \end{itemize}
            \end{coloredblock}
    
            \begin{coloredblock}[iee][\texttt{\textbf{\textbackslash begin\{enumerate\}}}:][\footnotesize\centering][][4.5cm]
                \begin{enumerate}
                \item Item level 1
                    \begin{enumerate}
                        \item Item level 2
                        \begin{enumerate}
                            \item Item level 3
                        \end{enumerate}
                    \end{enumerate}
                \end{enumerate}
            \end{coloredblock}
        \end{column}
        
        \begin{column}{0.49\textwidth}
            \begin{coloredblock}[green][\texttt{\textbackslash begin\{tugitemize\}} (tighter spacing):][\footnotesize\centering][][4.5cm]
                \vspace{0.5cm}
                \begin{tugitemize}
                    \item Item level 1
                    \begin{tugitemize}
                        \item Item level 2
                        \begin{tugitemize}
                            \item Item level 3
                        \end{tugitemize}
                    \end{tugitemize}
                \end{tugitemize}
            \end{coloredblock}
    
            \begin{coloredblock}[grey][\texttt{\textbackslash begin\{tugenumerate\}} (tighter spacing):][\footnotesize\centering][][4.5cm]
                \vspace{0.5cm}
                \begin{tugenumerate}
                    \item Item level 1
                    \begin{tugenumerate}
                        \item Item level 2
                        \begin{tugenumerate}
                            \item Item level 3
                        \end{tugenumerate}
                    \end{tugenumerate}
                \end{tugenumerate}
            \end{coloredblock}
        \end{column}
    \end{columns}
\end{frame}


\begin{frame}{Lists}
    \framesubtitle{Different Styles of Lists}

    \begin{coloredblock}[yellow]
        \centering
        There are \textbf{different styles} of \textbf{lists} that you can choose from.
    \end{coloredblock}

    \vspace{-0.5cm}

    % Bottom blocks (side-by-side)
    \begin{columns}
        \begin{column}{0.49\textwidth}
            \begin{coloredblock}[turquoise][\texttt{\textbf{\textbackslash begin\{romanenumerate\}}}:][\footnotesize\centering][][4.5cm]
                \begin{romanenumerate}
                    \item Item level 1
                    \begin{romanenumerate}
                        \item Item level 2
                        \begin{romanenumerate}
                            \item Item level 3
                        \end{romanenumerate}
                    \end{romanenumerate}
                \end{romanenumerate}
            \end{coloredblock}
    
            \begin{coloredblock}[yellow][\texttt{\textbf{\textbackslash begin\{boxenumerate\}}}:][\footnotesize\centering][][4.5cm]
                \begin{boxenumerate}
                \item Item level 1
                    \begin{boxenumerate}
                        \item Item level 2
                        \begin{boxenumerate}
                            \item Item level 3
                        \end{boxenumerate}
                    \end{boxenumerate}
                \end{boxenumerate}
            \end{coloredblock}
        
        \end{column}
        \begin{column}{0.49\textwidth}
            \begin{coloredblock}[blue][\texttt{\textbackslash begin\{tugromanenumerate\}} (tighter spacing):][\footnotesize\centering][][4.5cm]
                \vspace{0.5cm}
                \begin{tugromanenumerate}
                    \item Item level 1
                    \begin{tugromanenumerate}
                        \item Item level 2
                        \begin{tugromanenumerate}
                            \item Item level 3
                        \end{tugromanenumerate}
                    \end{tugromanenumerate}
                \end{tugromanenumerate}
            \end{coloredblock}
    
            \begin{coloredblock}[red][\texttt{\textbackslash begin\{boxromanenumerate\}}:][\footnotesize\centering][][4.5cm]
                    \begin{boxromanenumerate}
                        \item Item level 1
                        \begin{boxromanenumerate}
                            \item Item level 2
                            \begin{boxromanenumerate}
                                \item Item level 3
                            \end{boxromanenumerate}
                        \end{boxromanenumerate}
                    \end{boxromanenumerate}
            \end{coloredblock}
        \end{column}
    \end{columns}
\end{frame}


%% -------------------------------------------------------------------
%% Acknowledge Work with Proper Citations
\section{\textbf{Acknowledge} Work with Proper \textbf{Citations}}

\begin{frame}
    \agenda{Agenda}{} % No subtitle and highlights currentvsection
    %\agenda[sections=all]{Agenda}{Overview of lecture structure} % With subtitle and all sections highlighted
\end{frame}


\begin{frame}{References}
    \framesubtitle{Cite the Sources you use}

    \begin{coloredblock}[grey]
        \centering
        ”\textit{In the middle of every difficulty lies opportunity.}“
          
        \vspace{0.7cm}
        \scriptsize A. Einstein \cite{einstein2018}
    \end{coloredblock}
    
    \begin{coloredblock}[blue][\centering\texttt{\textbackslash textcite\{\}}]
        If you want to include the authors, use \texttt{\textbackslash textcite\{\}}.
    
        \vspace{0.5cm}
        In their study, \textbf{\textcite{gaugl2023}} demonstrate that rising CO\textsubscript{2} prices result in higher electricity prices in Austria.
    \end{coloredblock}
    
    \begin{coloredblock}[yellow][\centering\texttt{\textbackslash cite\{\}}]
        A simple reference is accomplished with \texttt{\textbackslash cite\{\}}.
    
        \vspace{0.5cm}
        The LEGO model is an energy system optimization model developed at the Institute of Electricity Economics and Energy Innovation. \textbf{\cite{wogrin2022}}
    \end{coloredblock}

\end{frame}


%% -------------------------------------------------------------------
%% Explore Design Inspirations to Spark  your Creativity
\section{Explore \textbf{Design Inspirations} to Spark  your Creativity}

\sectionheader{Design Inspirations}


\begin{frame}{Methodology}
    \framesubtitle{NTC Model}

    \begin{minipage}[t]{0.2\textwidth}
        \begin{coloredblockdark}[blue][][][][2.35cm][][0ex][0ex]
                \begin{minipage}[c][2.35cm][c]{0.4\textwidth}
                    \centering
                    \Large{\faIcon{coins}}
                \end{minipage}%
                \hfill
                \begin{minipage}[c][2.35cm][c]{0.55\textwidth}
                    \centering
                    \tiny\textbf{Objective Function:}\\
                    Minimize total system costs
                \end{minipage}%
                \hspace{0.05\textwidth}
        \end{coloredblockdark}
        \vspace{-0.55cm}
        \begin{coloredblockdark}[blue][][][][2.35cm][][0ex][0ex]
                \begin{minipage}[c][2.35cm][c]{0.4\textwidth}
                    \centering
                    \Large{\faIcon{balance-scale}}
                \end{minipage}%
                \hfill
                \begin{minipage}[c][2.35cm][c]{0.55\textwidth}
                    \centering
                    \tiny\textbf{Constraint 1:}\\ 
                    Balance equation
                \end{minipage}%
                \hspace{0.05\textwidth}
        \end{coloredblockdark}
        \vspace{-0.55cm}
        \begin{coloredblockdark}[blue][][][][2.35cm][][0ex][0ex]
                \begin{minipage}[c][2.35cm][c]{0.4\textwidth}
                    \centering
                    \Large{\faIcon{hand-paper}}
                \end{minipage}%
                \hfill
                \begin{minipage}[c][2.35cm][c]{0.55\textwidth}
                    \centering
                    \tiny\textbf{Constraint 2:}\\ 
                    NTC Limits
                \end{minipage}%
                \hspace{0.05\textwidth}
        \end{coloredblockdark}
        \vspace{-0.55cm}
        \begin{coloredblockdark}[blue][][][][2.35cm][][0ex][0ex]
                \begin{minipage}[c][2.35cm][c]{0.4\textwidth}
                    \centering
                    \Large{\faIcon{arrows-alt-v}}
                \end{minipage}%
                \hfill
                \begin{minipage}[c][2.35cm][c]{0.55\textwidth}
                    \centering
                    \tiny\textbf{Constraint 3:}\\ 
                    Flow Direction
                \end{minipage}%
                \hspace{0.05\textwidth}
        \end{coloredblockdark}
        \vspace{-0.55cm}
        \begin{coloredblockdark}[blue][][][][2.35cm][][0ex][0ex]
                \begin{minipage}[c][2.35cm][c]{0.4\textwidth}
                    \centering
                    \Large{\faIcon{arrows-alt-h}}
                \end{minipage}%
                \hfill
                \begin{minipage}[c][2.35cm][c]{0.55\textwidth}
                    \centering
                    \tiny\textbf{Constraint 4:}\\ 
                    Export/Import
                \end{minipage}%
                \hspace{0.05\textwidth}
        \end{coloredblockdark}
        \vspace{-0.55cm}
        \begin{coloredblockdark}[blue][][][][2.35cm][][0ex][0ex]
                \begin{minipage}[c][2.35cm][c]{0.4\textwidth}
                    \centering
                    \Large{\faIcon{power-off}}
                \end{minipage}%
                \hfill
                \begin{minipage}[c][2.35cm][c]{0.55\textwidth}
                    \centering
                    \tiny\textbf{Constraint 5:}\\ 
                    Generator Limits
                \end{minipage}%
                \hspace{0.05\textwidth}
        \end{coloredblockdark}
    \end{minipage}
    \hfill
    \begin{minipage}[t]{0.78\textwidth}
        \begin{coloredblock}[blue][][][][2.35cm][][0ex][0ex]
            \vspace{0.35cm}
            \begin{center}
                \small$\min \sum_{g,h} c_{g}^{op} \cdot p_g^{h} + \sum_{g} x_g \cdot (c_{g}^{Inv,MW} + c_{g}^{Inv,MWh} \cdot E2P_g)$
            \end{center}
        \end{coloredblock}
        \vspace{-0.55cm}
        \begin{coloredblock}[blue][][][][2.35cm][][0ex][0ex]
            \vspace{0.35cm}
            \begin{center}
                \small$\sum_{g} p_{gz(g,z),h} - \sum_{g} p_{cs(g,z),h} - \sum_{z \neq y} exp_{z,y,h} + \sum_{z \neq y} imp{z,y,h} = dem_{z,h}$
            \end{center}
        \end{coloredblock}
        \vspace{-0.55cm}
        \begin{coloredblock}[blue][][][][2.35cm][][0ex][0ex]
            \vspace{0.35cm}
            \begin{center}
                \small$exp_{z,y,h} \leq NTC_{z,y} \cdot bn_{z,y,h}$
            \end{center}
        \end{coloredblock}
        \vspace{-0.55cm}
        \begin{coloredblock}[blue][][][][2.35cm][][0ex][0ex]
            \vspace{0.35cm}
            \begin{center}
                \small$imp_{z,y,h} \leq NTC_{y,z} \cdot (1-bn_{z,y,h})$
            \end{center}
        \end{coloredblock}
        \vspace{-0.55cm}
        \begin{coloredblock}[blue][][][][2.35cm][][0ex][0ex]
            \vspace{0.35cm}
            \begin{center}
                \small$exp_{z,y,h} = imp_{y,z,h}$
            \end{center}
        \end{coloredblock}
        \vspace{-0.55cm}
        \begin{coloredblock}[blue][][][][2.35cm][][0ex][0ex]
            \vspace{0.35cm}
            \begin{center}
                \small$\underline{p_g} \leq p_{g,h} \leq \overline{p_g}$
            \end{center}
        \end{coloredblock}
    \end{minipage}
    
\end{frame}


\begin{frame}{Examined Main Scenarios}
    \framesubtitle{Analysis of Flexibility Requirements}

    \vspace{-.9cm}
    \begin{columns}
        \begin{column}{0.32\textwidth}
            \begin{coloredblock}[green][Scenario\\LOW ENERGY][\footnotesize\centering ]
                \begin{minipage}[t][3.5cm]{0.9\textwidth} 
                    \scriptsize STORYLINE: Austria continues to face a \textbf{challenging economic situation}, with \textbf{slowed investments} and \textbf{subdued economic growth}:
                \end{minipage}
                \begin{minipage}[t][2cm]{0.9\textwidth}
                    \begin{itemize}
                        \item \scriptsize \textbf{Investments} in \textbf{renewable energy} remain \textbf{low}, and \textbf{electricity demand} is \textbf{lower} than projected in the \textbf{ÖNIP} scenario.

                    \end{itemize}
                \end{minipage}
                \begin{minipage}[t][3.7cm]{0.9\textwidth} 
                    \scriptsize ~
                \end{minipage}
            \end{coloredblock}
        \end{column}
        \begin{column}{0.32\textwidth}
            \begin{coloredblock}[yellow][Scenario\\BASE][\footnotesize\centering ]
                \begin{minipage}[t][3.5cm]{0.9\textwidth} 
                    \scriptsize STORYLINE: Austria pursues a \textbf{highly ambitious path} to \textbf{decarbonization} and achieves the goal of decarbonization \textbf{by 2040}:
                \end{minipage}
                \begin{minipage}[t][2cm]{0.9\textwidth}
                    \begin{itemize}
                        \item \scriptsize Ambitious renewable energy expansion \textbf{aligned with the ÖNIP} (Austrian National Energy and Climate Plan).
                    \end{itemize}
                \end{minipage}
                \begin{minipage}[t][3.7cm]{0.9\textwidth} 
                    \scriptsize ~
                \end{minipage}
            \end{coloredblock}
        \end{column}
        \begin{column}{0.32\textwidth}
            \begin{coloredblock}[blue][Scenario\\ HIGH ENERGY][\footnotesize\centering ]
                \begin{minipage}[t][3.5cm]{0.9\textwidth} 
                    \scriptsize STORYLINE: \textbf{Investments} in \textbf{renewable energy exceed expectations}, accompanied by a \textbf{rapid pace} of \textbf{electrification}:
                \end{minipage}
                \begin{minipage}[t][2cm]{0.9\textwidth}
                    \begin{itemize}
                        \item \scriptsize This leads to \textbf{increased investments} in \textbf{renewables} and a \textbf{rising electricity demand}.
                    \end{itemize}
                \end{minipage}
                \begin{minipage}[t][3.7cm]{0.9\textwidth} 
                    \scriptsize \textit{The analysis examines how the extreme weather event impacts the generation fleet, storage requirements, transmission corridors, and the overall systemic behavior.}
                \end{minipage}
            \end{coloredblock}
        \end{column}
    \end{columns}

    \vspace{0.2cm}
    \begin{coloredblock}[grey]
        \centering
        \footnotesize\textbf{Through the in-depth analysis of the BASE scenario and the comparison with the extreme scenarios, the research questions are answered.}
    \end{coloredblock}

\end{frame}


\begin{frame}{Grading}

    \begin{coloredblock}[turquoise]
        \centering\footnotesize\textbf{Attendance mandatory\textsuperscript{1}}
    \end{coloredblock}

    \vspace{-0.5cm}
    \begin{columns}
        \begin{column}{0.32\textwidth}
            \begin{coloredblock}[blue][Participation][\footnotesize\centering][][6.5cm]
                \begin{itemize}
                    \item \footnotesize \textbf{Preparatory tasks}\\
                    Short tasks before class like watching a video, reading a text and answering some questions in TeachCenter
                    \item \footnotesize \textbf{Exercises during class}
                \end{itemize}
            \end{coloredblock}
            \centering \footnotesize \textbf{50 Point}
        \end{column}
        \begin{column}{0.32\textwidth}
            \begin{coloredblock}[blue][Homework 1][\footnotesize\centering][][6.5cm]
                \begin{itemize}
                    \item \footnotesize \textbf{Exercise}\\
                    Work you have to do at home.
                    \item \footnotesize \textbf{Report}\\
                    Comment your code and write a small report
                \end{itemize}
            \end{coloredblock}
            \centering \footnotesize \textbf{25 Point}
        \end{column}
        \begin{column}{0.32\textwidth}
            \begin{coloredblock}[blue][Homework 2][\footnotesize\centering][][6.5cm]
                \begin{itemize}
                    \item \footnotesize \textbf{Exercise}\\
                    Work you have to do at home.
                    \item \footnotesize \textbf{Report}\\
                    Comment your code and write a small report
                \end{itemize}
            \end{coloredblock}
            \centering \footnotesize \textbf{25 Point}
        \end{column}
    \end{columns}
    \vspace{0.7cm}
    \begin{coloredblock}[yellow]
        \centering\footnotesize\textbf{In each part more than 50\% of the points are required for a positive grade}
    \end{coloredblock}

    \vspace{0.2cm}
    \begin{coloredblock}[green]
        \begin{minipage}[c]{0.24\textwidth}
            \centering\footnotesize \textbf{Grading Scale}
        \end{minipage}
        \hfill
        \begin{minipage}[c]{0.24\textwidth}
            \footnotesize
            <50 Points:         5\\
            50 – 62 Points:     4
        \end{minipage}
        \hfill
        \begin{minipage}[c]{0.24\textwidth}
            \footnotesize
            63 – 75 Points:     3\\
            76 – 88 Points:     2
        \end{minipage}
        \hfill
        \begin{minipage}[t]{0.24\textwidth}
            \footnotesize
            89 – 100 Points:    1
        \end{minipage}
        
    \end{coloredblock}

    \addsource{\textsuperscript{1} If more than two lessons are missed without an adequate excuse, you can no longer successfully complete the course.}

\end{frame}


\begin{frame}{Classification}
    \framesubtitle{Definition}
    \small\centering\textbf{Classification} is a \textbf{machine learning technique} used to \textbf{predict group membership} for data instances.

    \vspace{.6cm}
    \begin{coloredblockicon}[yellow][Training Data][\footnotesize][\faIcon{table}]
        \vspace{.3cm}
        \begin{tugitemize}
            \item \scriptsize Given a collection of records (training set), each record contains a set of attributes, with one attribute being the class.
            \item \scriptsize Each sample is characterized by a tuple (X, y), where:
            \item \scriptsize X: attribute set (predictor, independent variable, input)
            \item \scriptsize y: class label (response, dependent variable, output)
        \end{tugitemize}
    \end{coloredblockicon}

    \vspace{0.1cm}

    \begin{coloredblockicon}[green][Task][\footnotesize][\faIcon{tasks}]
        \vspace{.3cm}
        \begin{tugitemize}
            \item \scriptsize Learn a model that maps each attribute set X to one of the predefined class labels y.
            \item \scriptsize Find a model for the class attribute as a function of the values of other attributes.
        \end{tugitemize}
    \end{coloredblockicon}

    \vspace{0.1cm}

    \begin{coloredblockicon}[blue][Goal][\footnotesize][\faIcon{bullseye}]
        \vspace{.3cm}
        \begin{tugitemize}
            \item \scriptsize Assign previously unseen records to a class as accurately as possible.
            \item \scriptsize Use a test set to determine the accuracy of the model.
            \item \scriptsize Typically, the dataset is divided into training and test sets:
            \item \scriptsize Training set: Used to build the model
            \item \scriptsize Test set: Used to validate the model
        \end{tugitemize}
    \end{coloredblockicon}

\end{frame}


\begin{frame}{Classification}
    \framesubtitle{Examples}

    \vspace{-1cm}
    \begin{columns}
        \begin{column}{0.32\textwidth}
        
            \begin{coloredblock}[yellow][\faIcon[regular]{smile-beam}~~~Sentiment Analysis][\footnotesize\centering][][6.2cm]
                \begin{itemize}
                    \item \scriptsize Task: Determine the sentiment of a text (e.g., reviews, social media posts).
                    \item \scriptsize Attributes: Text data (words, phrases).
                    \item \scriptsize Class Labels: "Pos," "Neg," or "Neutr"
                \end{itemize}
            \end{coloredblock}
            \vspace{0.1cm}
            \begin{coloredblock}[red][\faIcon{image}~~~Image Recognition][\footnotesize\centering][][6.2cm]
                    \begin{itemize}
                        \item \scriptsize Task: Identify objects or people in images.
                        \item \scriptsize Attributes: Pixel values, image features, etc.
                        \item \scriptsize Class Labels: "Cat," "Dog," etc.
                    \end{itemize}
            \end{coloredblock}
            
        \end{column}
        \begin{column}{0.32\textwidth}
            \begin{coloredblock}[blue][\faIcon{stethoscope}~~~Disease Diagnosis][\footnotesize\centering][][6.2cm]
                \begin{itemize}
                    \item \scriptsize Task: Diagnose whether a patient has a particular disease.
                    \item \scriptsize Attributes: Medical history, test results, symptoms.
                    \item \scriptsize Class Labels: "Disease" / "No Disease"
                \end{itemize}
            \end{coloredblock}
            \vspace{0.1cm}
            \begin{coloredblock}[turquoise][\faIcon{comments}~~~Language Detection][\footnotesize\centering][][6.2cm]
                \begin{itemize}
                    \item \scriptsize Task: Identify the language of a given text.
                    \item \scriptsize Attributes: Text data (words, characters).
                    \item \scriptsize Class Labels: "English," "Spanish," etc.
                \end{itemize}
            \end{coloredblock}
        
        \end{column}
        \begin{column}{0.32\textwidth}
        
        \begin{coloredblock}[green][\faIcon{running}~~~Activity Recognition][\footnotesize\centering][][6.2cm]
            \begin{itemize}
                \item \scriptsize Task: Determine the activity being performed based on sensor data.
                \item \scriptsize Attributes: Sensor readings (e.g., from a smartphone or wearable device).
                \item \scriptsize Class Labels: “Biking," "Running," etc.
            \end{itemize}
        \end{coloredblock}
        \vspace{0.1cm}
        \begin{coloredblock}[grey][\faIcon{envelope}~~~E-Mail Spam Detection][\footnotesize\centering][][6.2cm]
            \begin{itemize}
                \item \scriptsize Task: Classify as "spam" or "not spam."
                \item \scriptsize Attributes: Email content, sender, attachements etc.
                \item \scriptsize Class Labels: "Spam" or "Not Spam"
            \end{itemize}
        \end{coloredblock}
        
        \end{column}
    \end{columns}

\end{frame}


%% -------------------------------------------------------------------
%% Closing Slide
\section*{Closing Slide}

\begin{frame}
    \label{frame:closing_slide}
    % Insert closing slide
    \closingslide{Thank You!}{} % frametitle and subframetitle

    \addsource{Icons used in this presentation from Font Awesome (\url{https://fontawesome.com}), licensed under \href{https://creativecommons.org/licenses/by/4.0/}{CC BY 4.0}.}
\end{frame}


%% -------------------------------------------------------------------
%% Bibliography
\begin{frame}[allowframebreaks]{Bibliography}
  \printbibliography
\end{frame}

\end{document}